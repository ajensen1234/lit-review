While establishing a pipeline for the fully autonomous measurements of TKA kinematics, we encountered many of the different limitations present in using single-plane fluoroscopy. Fundamentally, this is a problem that exists inherently in the system, as you have a severely underconstrained problem, leading us to the inverse problem of computer vision (\cref{def:inverse-problem}).

\begin{mdframed}
    \begin{definition}[Inverse Problem]
        The inverse problem in computer vision is the process of calculating the causal factors (kinematics) the produced a set of observations (fluoroscopic images).
        \label{def:inverse-problem}
    \end{definition}
\end{mdframed}

One of the more pernicious limitations in single-plane fluoroscopy is an issue that we've dubbed the ``symmetry trap'', which causes multiple global minima when using a strictly contour-based objective function. The major contributor to these issues is symmetric tibial implants, which are mediolaterally symmetric (i.e. no different between right and left implants).

\begin{mdframed}
    \begin{definition}[Symmetry Trap]
        A symmetry trap occurs when a symmetric object has a projective geometry with more than one unique pose that can produce it. The simplest case is a sphere, where all poses produce the same circular projective geometry.
    \end{definition}
\end{mdframed}

This aim focuses on establishing a post-processing pipeline that can be used to address symmetry traps fully autonomously, solving an issue facing researchers studying single-plane kinematics for nearly 30 years! We also propose imaging recommendations to reduce the occurrence of symmetry traps in clinical data.

The work presented in this aim is currently under review for publication in the Journal of Biomechanics. This is some extra test text


%%% Local Variables:
%%% mode: latex
%%% TeX-master: "../../Andrew_Jensen_Dissertation"
%%% End:
