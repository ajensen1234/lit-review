The third aim of this thesis aims to express some of the non-publishable thoughts that the author has on his attempts to define a robust ``kinematics translator'', which would, for example, ``translate'' a patients ``stair rise'' kinematics into ``walking'' kinematics.
Additionally, the author explores many of the issues in the lack of standardization among different groups measuring TKA kinematics, and why that lack of standardization will make consistent clinical applications difficult.
He hopes this chapter is useful for others in this space to establish research protocols that can bring kinematics into both the clinical sphere and the ``big data'' world.

Ultimately, the hope is that there can be a standard operating procedure for measuring kinematics, such that the resultant measurements have post-operative predictive power with respect to function, stability, and modes of failure.


%%% Local Variables:
%%% mode: latex
%%% TeX-master: "../../Jensen-Lit-Review"
%%% End:
