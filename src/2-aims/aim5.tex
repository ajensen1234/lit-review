Parallel to clinical adoption of the Joint Track Machine Learning framework, we hope to expand the scope of use to research personnel studying joint kinematics around the world. Unfortunately, the current system is a system containing an exorbitant amount of legacy C++ code that requires extensive time and understanding to make changes. As such, the number of possibilities and ideas that can be tried and implemented remains in the hands of those students in the Gary J. Miller Ph.D. Orthopaedic Biomechanics Laboratory.

Thankfully, with the widespread adoption of Python as an easy-to-use language for researchers, and the ability to wrap back-end C++ subroutines with Python functions, it is possible to extend development of new and novel model-image registration pipelines to those without much software development expertise.

While working on specific research questions addressed above, current legacy code will slowly be replaced with dedicated and well documented libraries and subroutines, all of which will get Python wrappings. We hope that the introduction of this will allow other research groups to develop novel and unique model-image registration pipelines using the speed and accuracy of Joint Track Machine Learning as a foundation.

%%% Local Variables:
%%% mode: latex
%%% TeX-master: "../../Jensen-Lit-Review"
%%% End:
