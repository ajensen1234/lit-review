The successful application of JTML to TKA implants suggested potential applicability to other implant types and joints.
This hypothesis gained credence from JTML's consistent performance across diverse TKA implant styles, spanning posterior stabilized and cruciate retaining designs, as well as variations in the peg design of the tibial baseplate.
Consequently, the TKA experimentation framework was replicated, applying it to TSA implant data, to assess whether this success could be mirrored in a different joint context.
This section systematically categorizes the challenges encountered in adapting JTML for TSA, detailing each issue and its implications on the results.

\subsection{Methods}
The study utilized 823 post-operative reverse TSA (rTSA) images, complete with human-supervised kinematics, sourced from Nagoya University in compliance with IRB guidelines.
These images were obtained using single-plane fluoroscopy to measure glenohumeral kinematics in patients performing anonymized movements.
The collected data for each patient included deidentified radiographic images, imaging calibration files, and manufacturer-supplied glenoid and humeral implant surface geometry files.


The same CNN architecture previously used for TKA implants \cite{wangDeepHighResolutionRepresentation2020} (\cref{sec:jtml}) was employed, maintaining an 80/20 training/testing split.
A key consideration in the methodology was the limited availability of human-supervised rTSA images with kinematics compared to TKA.
To address this, the training set was augmented with non-affine transformations, specifically grid distortion and elastic transform \cite{buslaevAlbumentationsFastFlexible2020}.
While such transformations increased training times in the TKA pipeline, their computational intensity was manageable given the smaller rTSA image dataset, enhancing the network's generalization capability for rTSA image segmentation.

Next, the DIRECT-JTA algorithm from the JTML suite was applied to rTSA images, evaluating the autonomous kinematics measurement platform.
The efficacy of the algorithms will be assessed by comparing the root-mean-square difference between autonomous and human-supervised kinematic measurements on this novel test set.

\subsection{Results}
\begin{table}[h!]
	\caption{Root mean squared differences between JointTrack Machine Learning optimized kinematics and manually registered kinematics on single-plane fluoroscopy} \label{tab:jtml-tsa-tka-vals}
	\begin{tabularx}{\linewidth}{V{10cm}XXXXXX}\hline
		 Implant Type & $x_{trans} (mm)$ & $y_{trans} (mm)$ & $z_{trans} (mm)$ & $x_{rot} (^{\circ})$ & $y_{rot} (^{\circ})$ & $z_{rot} (^{\circ})$ \\ \hline
		Humeral            & 8.46             & 8.64             & 152.78           & 22.59                & 64.74                & 11.81                \\
		Glenosphere        & 0.97             & 1.44             & 32.58            & 13.72                & 26.40                & 8.30                 \\
		Femoral            & 0.57             & 0.39             & 26.95            & 0.66                 & 0.73                 & 0.60                 \\
		Tibial             & 0.67             & 0.64             & 27.17            & 1.63                 & 2.74                 & 0.66                 \\\hline
	\end{tabularx}
\end{table}


\subsection{Discussion}
The results obtained were suboptimal, defying the general intuition that a robust segmentation network would naturally lead to successful outcomes.
It appears this challenge is akin to the so-called ``symmetry trap'' encountered with tibial implants, where a symmetric axis in a 3D object presents difficulties in model-image registration optimization.
Broadly, the errors were categorized into two main groups.
The first category related to the internal/external rotation axis.
The semi-cylindrical shape of the humeral implant along the internal/external rotation axis is likely to decrease sensitivity to changes in shape during different rotation angles.
This decreased shape sensitivity, compounded by any imperfections in the segmentation, likely made it extremely difficult for the optimization routine to identify the global minima.
To address the issue of decreased shape sensitivity, a first-principles geometric approach to shape descriptors has been embarked upon, which will be elaborated upon in the Shape Sensitivity Analysis study (\cref{sec:shape-sensitivity}).
The second category involved the distal end of the humeral implant registering correctly, while the proximal articular surface did not align as expected. This issue may arise from the image similarity cost function's lack of preference for any specific region along the implant's contour, thereby allowing it to become trapped in a semi-reasonable local minima.
To mitigate the issue stemming from the cost function's non-preferential treatment of the contour, development of novel cost functions is underway, as will be discussed in the subsequent section (\cref{sec:new-image-metrics}).

%%% Local Variables:
%%% mode: latex
%%% TeX-master: "../../Andrew_Jensen_Dissertation"
%%% End:
