\section{Conclusion}
This study demonstrates intrinsic differences between implant types regarding projected 2D shape sensitivity.
Measurement difficulties aligned with low sensitivity along problematic axes—humeral internal rotation and tibial symmetry traps.
Fundamentally, small orientation changes yielded negligible 2D variability for near-symmetrical geometries and axes.
While inherent shape constraints limit data extractable solely from single-plane fluoroscopic silhouettes, incorporating additional image information like bone offers promise.
Despite unavoidable ambiguity along select dimensions, boosting descriptor sensitivity and employing precise anatomical constraints could enable robust clinical tracking.
Overall, relating optimization performance to shape response underscores routes toward accurate autonomous kinematic analysis.


%%% Local Variables:
%%% mode: latex
%%% TeX-master: "../../../Andrew_Jensen_Dissertation"
%%% End:
