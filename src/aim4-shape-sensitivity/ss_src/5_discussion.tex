\section{Discussion}
The findings correspond closely with initial expectations regarding the sensitivity measurement of projected shapes relative to 3D object orientation and are consistent with areas challenging for JTML optimization.
Specifically, the humeral implant showed a generally smooth and minimal shape sensitivity profile, particularly for $\delta_{y}$ rotations (\cref{tab:ss-vals}).
Along this axis, the humeral implant is the most cylindrical, meaning we would not expect to see a significant change in the shape descriptor with minor $\delta_{y}$ rotations.
Furthermore, it is noteworthy that this axis presented the most significant difficulties in JTML optimization.

Similar intuitive outcomes are observed with the glenosphere implant, which exhibited the lowest average $\mathbb{S}(\delta)$ among all the implant types.
This implant primarily consists of an articulation surface closely approximating a spherical shape.
Given that the projection of a sphere (a circle) remains constant regardless of the sphere's orientation, the closer a shape is to a spherical form, the lower its overall shape sensitivity is expected to be.

The observed shape sensitivity of the tibial implant with respect to $\delta_{y}$ rotation aligns with the concept of symmetry traps.
There is a consistently low shape sensitivity along the line where $x=0$.
This axis, associated with internal/external rotation, is the same one that contributes to symmetry traps, where two different 3D orientations result in an identical projected shape.
In terms of this analysis, the $\Delta S$ value would be $0$ for these two orientations of the tibial implant.

This study sheds light on an important aspect of JointTrack Machine Learning, particularly the use of Euler angles in the DIRECT-JTA optimization routine.
Currently, the optimization does not involve independently varying all angles within a body-centered reference frame, as this approach is not conducive to hyperbox creation \cite{floodAutomatedRegistration3D2018,jonesLipschitzianOptimizationLipschitz1993}.
Instead, optimization is performed over a range of ordered rotations, projected through the sequence $R_{z}R_{x}R_{y}$.
The challenges the humeral implant encounters in aligning the $y$-axis illustrate that this ordered sequence, especially with a symmetric final axis, can hinder the convergence process.

Beyond the inherent shape sensitivities, such optimization limitations motivate exploring alternatives to Euler angles.
Performing registration optimization directly on the Special Orthogonal group $SO(3)$ poses an intriguing direction.
$SO(3)$ encapsulates all possible 3D rotations in a mathematically convenient structure (A \emph{Lie Group}, which is both a manifold and a group) \cite{zillerLieGroupsRepresentation2010,serreLieAlgebrasLie1992}.
By optimizing on this manifold instead of Euler angle parametrizations, issues with gimbal lock and cascading rotation effects can be avoided.
However, most manifold optimization is specifically tailored to derivative-based optimization, which is not currently supported by the DIRECT-JTA algorithm that JointTrack Machine Learning incorporates \cite{jensenJointTrackMachine2023,jonesLipschitzianOptimizationLipschitz1993,floodAutomatedRegistration3D2018}.
This limitation motivates either a restructuring of the current optimization methods to incorporate derivative information, or to explore the potentials of derivative-free (black-box) optimization \cite{audetDerivativeFreeBlackboxOptimization2017} over manifolds.


Historical manual registration software heavily relied on bony landmarks within images to disambiguate challenging implant poses.
As landmarks, the tibial tuberosity, fibula, and bicepital groove were vital in identifying the specific orientation of implants, all of which are visible and utilized by the user in a manual registration setting (\cref{fig:TSA-multiview}).
To accurately assess implant kinematics in highly symmetric unicompartmental knee arthroplasty, surface meshes of femoral and tibial components were incorporated into the registration process \cite{banksComparingVivoKinematics2005}.
These bone models were particularly beneficial in addressing implants' internal/external rotation along semi-symmetric axes.
Additionally, the use of densely selected keypoints in the precise measurement of preoperative shoulder kinematics through biplane fluoroscopy has been documented \cite{burtonFullyAutomaticTracking2023}, with keypoint registration achieved via a modified Perspective-N-Points optimization routine.
Furthermore, intensity-based metrics have played a crucial role in the accurate determination of non-operative tibiofemoral kinematics \cite{bakaStatisticalShapeModelBased2012}.
Rather than minimizing the Euclidean distance of projected model keypoints, these methods leverage digitally reconstructed radiographs and directly compare the intensity values of the x-ray image and the projection estimate.
Thus, utilizing bony landmarks for measuring implant kinematics presents a robust and reliable approach to measuring accurate joint kinematics in clinical environments.



%%% Local Variables:
%%% mode: latex
%%% TeX-master: "../Jensen_Shape_Sensitivity"
%%% End:
