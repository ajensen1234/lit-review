\chapter{This Will Definitely Work on Shoulder Implants, Right?\protect\footnotemark}
\footnotetext{No}

\section{Introduction}
Another common arthroplasty (TSA) procedure is total shoulder replacement, which involves the removal and replacement of the distal end of the glenoid/scapula and the proximal end of the humeral head.
Researchers, implant manufacturers, and surgeons, driven by similar motivations for understanding post-operative Total Knee Arthroplasty (TKA) kinematics, show equal interest in examining post-operative TSA kinematics.
Despite differing from TKA outcomes, a significant portion of TSA patients report dissatisfaction, predominantly attributing it to mechanical limitations or instability.

For the past twenty years, researchers have explored both anatomical and reverse total shoulder arthroplasty kinematics \cite{kijimaVivo3dimensionalAnalysis2015,matsukiVivo3DAnalysis2014,matsukiDynamicVivoGlenohumeral2012,sugiComparingVivoThreedimensional2021,burtonFullyAutomaticTracking2023}, attempting to draw connections between different implants and surgical techniques to postoperative shoulder dynamics and range of motion.
For the same reasons that a reliable, fast, clinically practical, and fully-autonomous kinematics measurement platform would be desirable for clinicians operating on post-operative total knee arthroplasty (\cref{sec:jtml}), it would also be desirable for clinicians working on total shoulder arthroplasty.

Unfortunately, applying the same pipeline to shoulder implants did not work.
This dissertation chapter is divided into three sections.
First, I will discuss the methods and results of applying our existing framework to the problem of measuring TSA implant kinematics, and explain some of the pitfalls.
Second, I will dive into some of the modifications made to the model-image registration cost-function to try and overcome some of the TSA kinematics limitations.
Last, I will discuss a geometric first-principles approach used to dive even deeper into diagnosing the inherent kinematics measurement issues, and propose of methodological recommendations for others wishing to perform these measurements.
This last chapter will be converted into a paper for hopeful publication.

\section{Joint Track Machine Learning on Total Shoulder Arthroplasty Implants}
JTML on TSA


%%% Local Variables:
%%% mode: latex
%%% TeX-master: "../../Andrew_Jensen_Dissertation"
%%% End:

\section{Improving Model-Image Registration}
\label{sec:new-image-metrics}
Given the relatively poor performance of the standard optimization algorithm (DIRECT \cite{jonesLipschitzianOptimizationLipschitz1993,floodAutomatedRegistration3D2018}) and the cost function ($L_{1}$-distance or Hamming Distance \cite{floodAutomatedRegistration3D2018}), this study initially investigated the potential for a more robust image similarity representation to enhance performance (\cref{sec:image-similarity}).
The identification of global minima in this model-image registration challenge was facilitated by exploring several approaches to enhance image similarity representation.

The first approach involved attempts to enhance the convexity of the problem.
With the existing formulation, the cost was observed to reach maximum error ($\sum I + P$) when there was no overlap between the projection and the CNN-generated segmented image, regardless of the projection's deviation, whether it be 5 or 500 pixels.
This highlighted the need for a more nuanced error gradient.

Secondly, an exploration of the perceptual psychology of shape was undertaken \cite{attneaveInformationalAspectsVisual1954,attneaveQuantitativeStudyShape1956}.
Given that manual registration is considered the benchmark for ground-truth kinematics, an understanding of human perception of shape differences and overlaps was deemed crucial.
Consultations with surgeons and engineers were conducted, providing invaluable insights into refining the registration approach, with an emphasis on identifying effective procedures and features.

Exploring the addition of new cost parameters for JointTrack Machine Learning faced certain constraints.
The introduction of new cost functions was required to be non-conflicting with the Hamming Distance, necessitating their concurrent minimization with the Hamming Distance for all additional metrics.
Furthermore, the algorithm was required to support parallel processing in CUDA without significantly increasing the computational burden per image.
Lastly, maintaining the existing efficiency, characterized by the necessity for only one kernel per iteration to preserve application performance, was considered crucial.




\subsection{Improving Error Gradient}
For gradient enhancement, it is essential to establish notions of \emph{closeness} and \emph{farness} in the image plane, which transcends the basic \emph{hits} and \emph{misses} of the Hamming distance calculation.
Functions that satisfy this property and provide consistent minima with the Hamming distance are defined as \textit{surface distances} \cite{reinkeCommonLimitationsImage2023,reinkeUnderstandingMetricrelatedPitfalls2023}.
A surface distance represents the mathematical formulation which captures the concept of closeness or farness between contours in an image plane.
Within surface distances, a notable subcategory is \emph{symmetric surface distances}, characterized by the equality $d(a,b) = d(b,a)$; however, this symmetry is not a prerequisite for a metric to be useful.
Four primary metrics are widely recommended for the evaluation of surface distance \cite{reinkeUnderstandingMetricrelatedPitfalls2023,reinkeCommonLimitationsImage2023}.

The first is the Normalized Surface Distance or Normalized Surface Dice (\Cref{fig:surfDICE}) \cite{nikolovClinicallyApplicableSegmentation2021} .
However, surface DICE encounters a similar limitation as the Hamming distance, being maximized at points of no overlap, without discerning between closer and farther estimations.

\begin{figure}[h!]
  \centering
  \includegraphics[width=0.9\textwidth]{~/figures/raster/NSD.png}
  \caption{A graphical representation of the Normalized Surface Distance from \cite{reinkeUnderstandingMetricrelatedPitfalls2023,reinkeCommonLimitationsImage2023}.}
  \label{fig:surfDICE}
\end{figure}

The second is the Mean Average Surface Distance or Mean Surface Distance (\Cref{fig:MASD}) \cite{benesPerformanceEvaluationImage2015}.
This metric calculates the average of the mean shortest distances from every point on one boundary to any point on the other boundary \cite{reinkeCommonLimitationsImage2023,reinkeUnderstandingMetricrelatedPitfalls2023}.
A significant drawback of this method lies in its requirement to spawn a sub-kernel for each sampled pixel on the target and estimated contour, substantially increasing the computational load.

\begin{figure}[h!]
  \centering
  \includegraphics[width=0.9\textwidth]{~/figures/raster/MASD.png}
  \caption{A graphical representation of the Mean Average Surface Distance from \cite{reinkeCommonLimitationsImage2023,reinkeUnderstandingMetricrelatedPitfalls2023}.}
  \label{fig:MASD}
\end{figure}

The third is the Average Symmetric Surface Distance (\Cref{fig:ASSD}) \cite{yeghiazaryanFamilyBoundaryOverlap2018}.
This metric represents a symmetric variation of the mean average surface distance, calculating the average distance from each contour to the other, rather than the mean of the average shortest distances.
Similar to the Mean Average Surface Distance, this metric also necessitates sub-kernels for each sampled point, significantly increasing the computational requirements per iteration.

\begin{figure}[h!]
  \centering
  \includegraphics[width=0.9\textwidth]{~/figures/raster/ASSD.png}
  \caption{A graphical representation of the Average Symmetric Surface Distance from \cite{reinkeUnderstandingMetricrelatedPitfalls2023,reinkeCommonLimitationsImage2023}.}
  \label{fig:ASSD}
\end{figure}

The last is the Hausdorff Distance (\Cref{fig:HD}) \cite{huttenlocherMultiresolutionTechniqueComparing1993,felzenszwalbDistanceTransformsSampled2012,huttenlocherComparingImagesUsing1993}.
The Hausdorff distance is the maximum distance from a point on one boundary to the nearest point on another boundary. Typically the Hausdorff distance for an entire contour is taken as the average of the Hausdorff distances for a series of sampled points on the target contour.
Like the previous two, the Hausdorff distance requires sub-kernels for every sampled point, and so it is not feasible.


\begin{figure}[h!]
  \centering
  \includegraphics[width=0.9\textwidth]{~/figures/raster/HausdorfDistance.png}
  \caption{A graphical representation of the Hausdorff Distance, taken from \cite{reinkeCommonLimitationsImage2023,reinkeUnderstandingMetricrelatedPitfalls2023}.}
  \label{fig:HD}
\end{figure}

At this point, none of the recommended distance metrics were found to satisfy the criteria for a feasible cost function capable of introducing an error gradient.
Based on the limitations identified in each approach, the pre-computation of as many values as possible to reduce the algorithmic load during optimization was considered ideal.
In previous studies, 3D distance maps were pre-computed to facilitate medical model-image registration \cite{lavalleeRecoveringPositionOrientation1995,zuffiModelbasedMethodReconstruction1999}.
Consequently, a cost function was devised that utilizes pre-computed distance maps to introduce an error gradient, without the need to spawn multiple kernels during each optimization iteration.



\subsection{Modified Mean Distance Cost Function}
Initially, rather than calculating the average distance across all target points – a process that would necessitate spawning multiple sub-kernels – a distance map encoding the distance to the \emph{nearest point} on the target contour was introduced.


With an arbitrary image point defined as $p_{xy}$, and the target contour defined as $T$, this distance map can be expressed as a grid, $\displaystyle DM_{xy}(T) = \min_{t\in T}d(p_{xy},t)$, where $d(p_{xy},t)$ represents any chosen distance function.
In this case, the $L_{1}$-distance was selected for efficient computation, utilizing OpenCV's \texttt{distanceTransform()} function for this purpose \cite{bradskiOpenCVLibrary2000}.

Subsequently, a notion of distance between the projected contour and the target contour can be expressed by calculating the average of the element-wise multiplication between the projection and the pre-calculated distance map (\cref{eq:DMCF}).
This approach offers the significant advantage of requiring only a single kernel for each iteration (iterating once per projection and performing a multiplication and atomic addition in memory) as well as sharing a minimum with the Hamming distance.
This is evident by observing that $DM_{x,y}=0$ for points on the target contour, indicating that if $Proj_{x,y}$ were perfectly aligned with this contour, the summation would result in the addition of zeros.


\begin{equation}
  \label{eq:DMCF}
  J = \dfrac{ \sum_{(x,y) \in \text{Image}} Proj_{x,y}DM_{x,y} }{\sum_{(x,y)\in \text{Image}}Proj_{x,y}}
\end{equation}

Unfortunately, the results of this endeavor did not improve the overall performance of the DIRECT algorithm in finding a global minimum for rTSA implants.
The performance was found to be very similar to the previously poor performance when using the Hamming distance cost function.

\subsection{The Psychology of Shape and Mimicking Human Operators}
Early research in computer vision was closely tied to the psychology and neurology of human perception.
Humans possess a remarkable ability to describe what they see, either through language or mathematical notation.
The task of programmatically replicating this skill in computers continues to present significant challenges.
Although some multi-modal speech/vision models have shown promise in a general sense, they lack a deep technical understanding of the visual content they process.


Historically, the first attempts to mathematically describe shapes, vision, images, and curves were rooted in psychological literature \cite{attneaveInformationalAspectsVisual1954,attneaveQuantitativeStudyShape1956,koenderinkStructureImages1984,koenderinkSurfaceShapeCurvature1992}.
This foundation appears appropriate for enhancing rTSA model-image registration.
The task essentially involves decomposing a contour into its basic shape and elemental components.

In the mid-1950s, psychologist Fred Attneave made significant contributions to the understanding of visual information redundancy and the identification of regions in images and shapes that humans perceive as highly salient \cite{attneaveQuantitativeStudyShape1956,attneaveInformationalAspectsVisual1954}.
He discovered that humans could recognize most images even when the shapes were significantly simplified, retaining only the most relevant features.
Attneave identified regions of high curvature, areas with a relatively high change in the normal vector, as carrying the most information about a shape (\cref{fig:attneave}).
He observed that most other line segments were superfluous for shape recognition.


\begin{figure}[h!]
  \centering
  \includegraphics[width=0.45\textwidth]{~/figures/raster/attneave-blob-points.png}
  \includegraphics[width=0.45\textwidth]{~/figures/raster/attneave-cat-contour.png}
  \caption{Some representative examples of the experiments that Fred Attneave performed to establish the primacy of high curvature as salient in shape recreation and recognition, from \cite{attneaveInformationalAspectsVisual1954}.}
  \label{fig:attneave}
\end{figure}

This understanding resonates with surgeons and engineers experienced in manual registration of rTSA implants.
They report that transitioning an image frame from “good” to “great” hinges on achieving visually precise alignment between the corners and edges of the projected 3D model and the fluoroscopic image.
Their emphasis on the importance of accurately aligning high-curvature regions aligns with Attneave's findings.


\begin{figure}[h!]
  \centering
  \includegraphics[width=0.3\textwidth]{~/figures/raster/TSA_original.png}
  \includegraphics[width=0.3\textwidth]{~/figures/raster/TSA_transparent.png}
  \includegraphics[width=0.3\textwidth]{~/figures/raster/TSA_solid.png}
  \caption[Some different model views of a manually registered humeral and glenoid implant in an rTSA system]{Some different model views of a manually registered humeral and glenoid implant in an rTSA system. Of note, each view gives the user a different type of feature to focus on. The original view allows the user to determine the relative orientation based on shading, the transparent view allows the user to see the underlying fluoroscopic image, and the solid view allows the user to focus on specific regions of error. Each is crucial to performing manual registration.}
  \label{fig:TSA-multiview}
\end{figure}

\subsection{Developing a Cost Function for Aligning High-Curvature Regions}

Informed by these findings, the aim was to develop a cost function that facilitates the alignment of high-curvature regions between the target shape and the projected shape.
The constraints, as outlined in the previous chapter, remained: any new cost functions must reach a global minimum concurrently with the Hamming distance, and computational efficiency is paramount.

Given that there is a finite number of high-curvature regions for any given implant, and that their total count can be limited, the disadvantages associated with many surface distances appeared mitigatable.
This mitigation strategy would involve focusing exclusively on these high-curvature regions rather than every pixel along the surface.
However, this approach introduced the challenge of having to spawn sub-kernels for each iteration.

To circumvent this, an \emph{asymmetric surface distance}, where $d(a,b) \ne d(b,a)$, became necessary.
Such a distance metric would allow for the pre-computation of a distance map based on the target shape.
Since this target does not change with each iteration, and memory usage is manageable due to the small number of points required for distance calculation, this approach seemed promising.
By focusing on pre-computed distances for critical high-curvature points, it was possible to maintain computational efficiency while potentially improving the accuracy of alignment.


\subsubsection{Finding regions of high curvature}
The first step in implementing this cost function involves determining the regions with high curvature in the contour, a task that is not as straightforward as it may appear.
This requires a contour-extraction algorithm, a discrete curvature equation, and a method for automatically selecting regions of high curvature.

OpenCV's \texttt{findContours} function \cite{bradskiOpenCVLibrary2000} is utilized, which provides a pre-specified number of contiguous contour points following an algorithm proposed by Suzuki and Abe \cite{suzukiTopologicalStructuralAnalysis1985}.
For the implementation at hand, 200 contour points are extracted.


For a discrete implementation of curvature, Menger's Algorithm \cite{legerMengerCurvatureRectifiability1999} is implemented.
This method defines the discrete curvature as the reciprocal of the radius of a circle fit through three points along the contour (\cref{eq:menger}).
Typically, a window of size $t$ is defined to increase the proximity of the three points along the curvature, making the calculations more robust to noise.
In this implementation, $t=18$.


\begin{equation}
  \label{eq:menger}
  \begin{split}
  C_{i} &= \dfrac{1}{radius(p_{i-t},p_{i},p_{i+t})}\\
        &\text{where} \\
        radius(x,y,z) &= \dfrac{4 \cdot Area}{|x-y||y-z||z-x|}
  \end{split}
\end{equation}

Given the utilization of neural network segmentation for the contour, the presence of some noise in the contour (\cref{fig:tsa-curv}) is inevitable, leading to noise in the curvature calculations.
Noise was mitigated by applying a 1-D Gaussian convolution to the array of curvature values, employing a 9-wide Gaussian kernel.
Regions of the highest curvature were then represented as peaks along the plot (\cref{fig:tsa-curv}).

To programmatically extract regions with high curvature, two steps were undertaken.
Initially, curvature values were filtered to exclude those where $c_{i} \le \mu_{curvature} + 1.5\sigma_{curvature}$, ensuring the selection of only high curvature points.
Subsequently, inflection points of the first derivative were identified to locate regions where the curvature transitioned from a positive to a negative derivative, indicative of the ``peak'' of that curvature region.
The points identified by this algorithm correspond precisely to the points where the specific contour exhibits the highest curvature, aligning with the regions that human operators intuitively identified as high curvature areas.


\begin{figure}[h!]
  \centering
  \includegraphics[width=0.9\textwidth]{~/figures/raster/TSA_curvature.png}
  \caption{A plot demonstrating the values of the Menger Curvature along the contour of an rTSA humeral implant. The regions of high curvature are the peaks of the plot.}
  \label{fig:tsa-curv}
\end{figure}



\subsubsection{Modified asymmetric surface distance}
Upon establishing a set of keypoints representing the regions of the highest curvature, the next step involved determining a pre-computable distance metric for creating a cost function.
The Hausdorff distance was ruled out due to its focus on the maximum distance point rather than incorporating each point's contribution to the overall distance.
The Symmetric distance was also dismissed in favor of seeking an asymmetric function to avoid the necessity of spawning sub-kernels during iteration calls.
Thus, the Mean Surface Distance was selected.
The modified approach eliminated distance calculations from the projection to the target (creating asymmetry) and focused only on distances centered at the selected keypoints.
Distance maps to each keypoint were pre-computed using OpenCV's \texttt{distanceTransform()}, and element-wise multiplication was applied to identify the nearest projected point to each keypoint (\cref{eq:curv-keypoint}).

\begin{equation}
  \label{eq:curv-keypoint}
  \begin{split}
    \displaystyle J &= \dfrac{\sum_{k \in \mathbb{K}}(\min_{p\in Proj}(p \cdot DM_{k}))}{N} \\
      &\text{where}\\
    \mathbb{K} &= \text{Set of all keypoints} \\
    N &= \text{Number of keypoints} \\
    DM_{k} &= \text{Distance map for keypoint $k$} \\
    p &= \text{Single point on projection silhouette}
  \end{split}
\end{equation}

Unfortunately, this approach also resulted in suboptimal performance.
Despite algorithmically identifying salient image features and minimizing the distance of these points to the estimation, the optimization routine failed to accurately match the solution to human-supervised registration.

\subsection{Discussion}
This section delved into the exploration of applying JointTrack Machine Learning to reverse Total Shoulder Arthroplasty implants.
Although the applications did not yield a successful solution for rTSA model-image registration, they contributed to expanding the body of knowledge in the field.

First, all algorithms were rigorously tested on TKA implant registration to ensure no adverse impact on the established performance of JTML.
Given the strengths listed for each algorithm, they might offer stronger and more robust performance than the traditional Hamming distance for TKA implants.
This hypothesis warrants further testing and evaluation, suggesting that the described additions could significantly improve JTML's convergence speed and overall accuracy.

Moreover, although the methods did not succeed in aligning rTSA implants, they demonstrated a comprehensive understanding and a valuable resource for model-image registration endeavors.
The methodical and calculated approach to employing a wide array of mathematical tools to tackle a challenging problem is beneficial for anyone facing a similar technical obstacle.

Lastly, the inability of the methods to accurately perform model-image registration highlighted potential deficiencies in the overall pipeline.
One such deficiency could be attributed to the performance ceiling of the CNN segmentation, though this seems unlikely given the state-of-the-art IOU scores.
Incorporating a surface-based metric into the neural network error functions might yield a more robust segmentation \cite{reinkeCommonLimitationsImage2023,reinkeUnderstandingMetricrelatedPitfalls2023}.
Another limitation could be the use of Euler angles in the optimization process, where a more robust rotation parametrization, such as Quaternions, might mitigate the non-commutativity issues.
Additionally, the shape of humeral implants compared to TKA implants, which are more ``cylindrical,'' suggests that minor orientation changes do not significantly alter the projected shape, complicating optimization efforts (\Cref{sec:shape-sensitivity}).

Each identified deficiency represents a promising avenue for future research and development.
The exploration of these potential areas of improvement could lead to significant advancements in model-image registration techniques.
The final section of this chapter will address the potential sensitivity of the projected shape to input orientation, further exploring the nuances of this complex and evolving area of study.


%%% Local Variables:
%%% mode: latex
%%% TeX-master: "../../Andrew_Jensen_Dissertation"
%%% End:

\section{2D Shape Projection Sensitivity Analysis}
\label{sec:shape-sensitivity}

Understanding the in-vivo kinematics of total joint replacement has been essential in implant design, post-operative assessment, and predicting wear and failure patterns for nearly three decades \cite{freglyComputationalWearPrediction2005,banks2003HapPaul2004,banksRationaleResultsFixedBearing2019}.
Recent advancements in computer vision and machine learning have enabled these analyses for total knee arthroplasty (TKA) in a fully autonomous and clinically practical setting, utilizing single-plane fluoroscopy \cite{brobergValidationMachineLearning2023,jensenJointTrackMachine2023}.
However, using only a single camera inherently limits the measurement accuracy due to loss of depth perception and the introduction of ambiguous projected shapes during optimization \cite{floodAutomatedRegistration3D2018,mahfouzRobustMethodRegistration2003,zuffiModelbasedMethodReconstruction1999,banksAccurateMeasurementThreedimensional1996}.
The observed limitation, predominantly impacting mediolaterally symmetric tibial implants, led to a phenomenon termed “symmetry traps.”
In such instances, two distinct three-dimensional orientations of the implant produce indistinguishable two-dimensional projected geometries.
A machine learning algorithm was developed to address these symmetry traps in symmetric tibial implants.
This algorithm was trained to recognize accurate anatomic orientations and correct images caught in optimization minima \cite{jensenCorrectingSymmetricImplantInReview}.
However, this approach required the symmetric implant to register into one of the two potential local minima, each corresponding to a distinct “symmetry trap.”

The application of the same optimization routine and cost function \cite{floodAutomatedRegistration3D2018,jensenJointTrackMachine2023} to reverse total shoulder arthroplasty (rTSA) resulted in significantly lower performance compared to its application in TKA implants (\cref{tab:jtml-tsa-tka-vals}) \cite{jensenJointTrackMachine2023}.
This suboptimal performance manifested primarily along the internal/external rotation axis, which has salient features often occluded by the glenosphere implant in frontal-plane fluoroscopy (\cref{fig:bad_ie_hum}).
Additionally, this axis is nearly rotationally symmetric for both the humeral and glenospere implants.
Poor rotation registration also increases translation errors, as the silhouette shape of the estimated pose is wholly different from the fluoroscopic image, causing imprecise translation alignment along all axes.
In a manual registration setting \cite{muJointTrackOpenSourceEasily2007}, different combinations of model and image views are utilized to overcome these limitations (\cref{fig:TSA-multiview}).

\begin{figure}[h!]
	\centering
	\includegraphics[width=0.4\textwidth]{~/figures/raster/BAD_IE_HUM.png}
	\caption[A representative example of poor internal/external rotation of the humeral implant after automated model-image registration using JointTrack Machine Learning]{A representative example of poor internal/external rotation of the humeral implant after automated model-image registration using JointTrack Machine Learning \cite{jensenJointTrackMachine2023}.}
	\label{fig:bad_ie_hum}
\end{figure}

The current investigation delves into the fundamental shape aspects of each arthroplasty system, with a focus on developing a method for autonomously measuring rTSA kinematics from single-plane fluoroscopy.
Central to this is the use of Invariant Shape Descriptors, particularly the Invariant Angular Radial Transform Descriptor (IARTD), which offers a mathematically robust approach to describe object shapes \cite{leeNewShapeDescription2012}.
These descriptors are immune to variations in scale, translation, or orientation \cite{zhangReviewShapeRepresentation2004}, and are adept at quantifying the relative ``nearness'', ``farness'', and ``uniqueness'' of shapes as vector differences.
Such properties are valuable for object categorization \cite{richardIdentificationThreeDimensionalObjects1974,wallaceAnalysisThreedimensionalMovement1980,wallaceEfficientThreedimensionalAircraft1980} and kinematics measurement \cite{banksAccurateMeasurementThreedimensional1996}, with IARTD's sensitivity to radial shape differences \cite{leeNewShapeDescription2012} being particularly beneficial for detailed contour analysis.

The focus of this analysis is on the sensitivity of projected 2D shapes, as depicted by IARTD, to changes in their 3D orientation.
This is key to understanding the impact of subtle orientation variations on the projected shape, an aspect integral to shape-based optimization metrics.
The ultimate aim is to highlight performance differences in autonomous kinematics measurements between TKA and rTSA implant systems.
Additionally, the study seeks to identify areas where imaging methods can be improved to boost the algorithm's performance.

%%% Local Variables:
%%% mode: latex
%%% TeX-master: "../../../Andrew_Jensen_Dissertation"
%%% End:

\section{Methods}


\subsection{Data Collection}
For shape sensitivity analysis, representative 3D models of rTSA humeral and glenosphere implants, as well as TKA femoral and tibial implants, were obtained from a manufacturer.
The study focused on a single size for each implant type, as the scale of the shapes was normalized using an Invariant Shape Descriptor, rendering multiple sizes unnecessary for this analysis.
\subsection{Image Generation}
Each implant's binary silhouette was rendered to a $1024\times 1024$ image plane using an in-house CUDA camera model (CUDA Version 12.1) \cite{nickollsScalableParallelProgramming2008}.
The model featured a 1000mm focal length and 0.3mm per pixel resolution, which are quite typical projection parameters for fluoroscopic images.
All imaging tasks utilized an NVIDIA Quadro P2200 GPU.
\subsection{Invariant Angular Radial Transform}
The Invariant Angular Radial Transform Descriptor (IARTD) was selected for its radial direction sensitivity, enabling the detection of subtle contour changes in projected shapes \cite{leeNewShapeDescription2012}.
This sensitivity allows us to address minor changes along the contour of the projected shape, which is a desirable property for determining the minor changes in shape with respect to input orientation.

IARTD computation involves aggregating orthogonal basis components across the unit polar disk, forming a complex moment.
Each basis function has an order ($n$) and a repetition ($p$).
The order can be visualized as concentric rings on the polar disk, and the repetition as the count of slices partitioning the unit disk along $\theta$.
For these calculations, the image is normalized so that the center is at $(0,0)$, and the four corners are at $(\pm1,\pm1)$.

Each angular radial transform (ART) coefficient is a complex double integral (\cref{eq:F_np}) over the image in polar coordinates, $f(\rho,\theta)$ multiplied by the ART basis function, $V_{np}(\rho,\theta)$ (\cref{eq:V_np}).

\begin{equation}
	\label{eq:F_np}
	F_{np} = \int_{0}^{2\pi}\int_{0}^{1}f(\rho,\theta)V_{np}(\rho,\theta)\rho d\rho d\theta
\end{equation}

\begin{equation}
	\label{eq:V_np}
	V_{np}(\rho,\theta) = A_{p}(\theta)R_{n}(\rho)
\end{equation}

The radial basis function includes a complex exponential, $A_{p}(\theta)$ (\cref{eq:A_p}), ensuring rotational invariance, and a trigonometric transform, $R_{p}(\theta)$ (\cref{eq:R_n}), to establish orthogonality.

\begin{equation}
	\label{eq:A_p}
	A_{p}(\theta) = \dfrac{1}{2\pi}e^{jp\theta}
\end{equation}
\begin{equation}
	\label{eq:R_n}
	R_{n}(\rho) =
	\begin{cases}
		1                   & n=0     \\
		2 \cos (\pi n \rho) & n \ne 0
	\end{cases}
\end{equation}

Phase correction is applied to each ART coefficient (\cref{eq:art_phase_correction}, \cref{eq:fnp_phase_correction}) to adjust for differences in in-plane rotation.

\begin{equation}
	\label{eq:art_phase_correction}
	\phi'_{np} = \phi_{np} - \phi_{n,1}
\end{equation}

\begin{equation}
	\label{eq:fnp_phase_correction}
	F_{np}' = F_{np}e^{-jp\phi_{n,1}}
\end{equation}

Subsequently, the final feature vector is formulated by the polar decomposition of each coefficient at every order and repetition (\cref{eq:iartd}).
Values from the first two repetitions are excluded, as they do not provide significant information \cite{leeNewShapeDescription2012}.
The complete IARTD feature vector encompasses values of $n={0, \dots, 3 }$ and $p={0, \dots, 8}$ per the original authors' suggestion \cite{leeNewShapeDescription2012}.

\begin{equation}
	\label{eq:iartd}
	IARTD = \{|F'_{np}|, \phi_{np}'\} \text{ where } n \ge 0, p \ge 2
\end{equation}

\subsection{Shape Differences and Sensitivity}
In order to quantify the overall change between shapes, a readily interpretable value must be established.
To simplify notation, successive rotations are denoted as subscripts, with $R_{z}R_{x}R_{y}$ being represented as $R_{z,x,y}$.
If more than 3 rotations are applied successively, the full rotation sequence will be captured as $R_{r_{1},r_{2},r_{3},\cdots,r_{n}}$.
Similarly, the application of the IARTD equation to an implant at a specific input orientation $R_{z,x,y}$ is denoted as $IARTD(R_{z,x,y})$.
Shape differences were calculated using the central difference equation on the IARTD vector produced from two different orientations.
The grid of sampled orientations had extrema of $\pm 30^{\circ}$ with a step size of $5^{\circ}$ for each of the $x$, $y$, and $z$ axes.
The ``differences'' along each axis were computed using a positive and negative rotation ($\pm \delta $) of 1 degree .
Therefore, for every set of $x,y,z$ rotations, three distinct shape differences are computed, one each for $\delta_{x}$, $\delta_{y}$, and $\delta_{z}$ (\cref{eq:shape-derivative}).

\begin{figure}[h!]
  \centering
  \includegraphics[width=0.5\textwidth]{~/figures/raster/rTSA_humeral_rotation_axes.png}
  \caption{A generic manufacturer-provided humeral implant with label x, y, and z rotation axis. Additionally, each of the ,$\delta_x$,$\delta_y$, and $\delta_z$ are shown, corresponding to the rotational directions of each shape descriptor difference.}
  \label{fig:rot-axis}
\end{figure}

For notational brevity, we will condense the total equation to a single $\Delta S(\delta)$, (representing $\Delta Shape$ for a differential rotation $\delta$).

\begin{equation}
	\label{eq:shape-derivative}
	\begin{split}
		\Delta S(\delta)_{z,x,y}  \equiv & IARTD(R_{z,x,y,+\delta})                        \\
		                                 & - IARTD(R_{z,x,y,-\delta})                      \\
		\forall                          & \delta \in \{\delta_{x},\delta_{y},\delta_{z}\}
	\end{split}
\end{equation}

The disparate scales of IARTD vector elements necessitate their normalization, ensuring a balanced assessment of global behavior without overemphasis on any individual element.
Z-scaling provides a practical approach to normalizing each element relative to its distribution.
After z-scaling, the Euclidean norm of each $S(\delta)_{z,x,y}$ is calculated to quantify the total shape change for a specific differential rotation (\cref{eq:euc_norm}).

The final step takes advantage of two factors: first, that the in-plane rotations are the first in the Euler sequence ($z$-axis), and second, that this type of rotation does not affect the in-plane shape.
For each $x$ and $y$ input rotation, an average is computed from values where $x$ and $y$ remain constant while $z$ varies (\cref{eq:z_rot_norm}). The final values are obtained from this equation, denoted by $\mathbb{S}$.
Individual plots were created for $\mathbb{S}_{x,y}$, corresponding to each differential rotation in $x$, $y$, and $z$, and for each of the four implant types.
An analysis of these plots were conducted to assess the performance of JTML optimization and to explore areas where low shape-sensitivity will pose significant challenges for registration-based optimization.

\begin{equation}
	\label{eq:euc_norm}
	\|S(\delta)_{z,x,y}\|_{2}
\end{equation}

\begin{equation}
	\label{eq:z_rot_norm}
	\mathbb{S}(\delta)_{x,y} = \dfrac{\sum_{z} \| S(\delta)_{z,x,y} \|_{2}}{N}
\end{equation}

%%% Local Variables:
%%% mode: latex
%%% TeX-master: "../../../Andrew_Jensen_Dissertation"
%%% End:

\section{Results}
The humeral implant exhibited the lowest mean $\mathbb{S}(\delta_{y})$ across all implant types (\cref{fig:hum_sensitivity_plot}) (\cref{tab:ss-vals}).
This rotation represents the final rotation in our Euler rotation sequence (Z-X-Y) and captures the internal/external rotation of the humeral implant.
Additionally, the surface plotted by the humeral shape sensitivity for all $\delta_{x,y,z}$ is much smoother than any of the other plots, demonstrating the relative lack of shape difference for a wide range of input orientations.
Several plots showed regions with relatively low sensitivity.
Specifically, the glenosphere's $\delta_{y}$ sensitivity along the $y=0$ axis (\cref{fig:sca_sensitivity_plot}) and the tibial implant's $\delta_{y}$ sensitivity along the $x=0$ axis (\cref{fig:tib_sensitivity_plot}).
The femoral implant had the highest average sensitivity ($\frac{\mathbb{S}(\delta_{x}) +\mathbb{S}(\delta_{y}) +\mathbb{S}(\delta_{z})  }{3}$) among all implant types .


\begin{table}
	\caption{Average projected-shape sensitivity values for each of the implant models.} \label{tab:ss-vals}
	\begin{tabularx}{\linewidth}{|X|X|X|X|}\hline
		{\bf Implant Type} & Average $\mathbb{S}(\delta_{x})$ & Average  $\mathbb{S}(\delta_{y})$ & Average $\mathbb{S}(\delta_{z})$ \\ \hline
		Humeral            & 8.83                             & 4.82                              & 7.08                             \\\hline
		Glenosphere        & 6.37                             & 6.22                              & 4.86                             \\\hline
		Femoral            & 6.88                             & 8.68                              & 4.93                             \\\hline
		Tibial             & 9.0                              & 5.52                              & 3.72                             \\\hline
	\end{tabularx}
\end{table}


\begin{figure}[h!]
	\centering
	\includegraphics[width=0.3\linewidth]{~/figures/raster/Humeral_dx_sensitivity.png}
	\includegraphics[width=0.3\linewidth]{~/figures/raster/Humeral_dy_sensitivity.png}
	\includegraphics[width=0.3\linewidth]{~/figures/raster/Humeral_dz_sensitivity.png}
	\caption{The $\mathbb{S}$ plot for a humeral implant for $\delta$ rotations along the x, y, and z axis, respectively.}
	\label{fig:hum_sensitivity_plot}
\end{figure}

\begin{figure}[h!]
	\centering
	\includegraphics[width=0.3\linewidth]{~/figures/raster/Glenosphere_dx_sensitivity.png}
	\includegraphics[width=0.3\linewidth]{~/figures/raster/Glenosphere_dy_sensitivity.png}
	\includegraphics[width=0.3\linewidth]{~/figures/raster/Glenosphere_dz_sensitivity.png}
	\caption{The $\mathbb{S}$ plot for a glenosphere implant for $\delta$ rotations along the x, y, and z axis, respectively.}
	\label{fig:sca_sensitivity_plot}
\end{figure}
\begin{figure}[h!]
	\centering
	\includegraphics[width=0.3\linewidth]{~/figures/raster/Femoral_dx_sensitivity.png}
	\includegraphics[width=0.3\linewidth]{~/figures/raster/Femoral_dy_sensitivity.png}
	\includegraphics[width=0.3\linewidth]{~/figures/raster/Femoral_dz_sensitivity.png}
	\caption{The $\mathbb{S}$ plot for a femoral implant for $\delta$ rotations along the x, y, and z axis, respectively.}
	\label{fig:fem_sensitivity_plot}
\end{figure}
\begin{figure}[h!]
	\centering
	\includegraphics[width=0.3\linewidth]{~/figures/raster/Tibial_dx_sensitivity.png}
	\includegraphics[width=0.3\linewidth]{~/figures/raster/Tibial_dy_sensitivity.png}
	\includegraphics[width=0.3\linewidth]{~/figures/raster/Tibial_dz_sensitivity.png}
	\caption{The $\mathbb{S}$ plot for a tibial implant for $\delta$ rotations along the x, y, and z axis, respectively.}
	\label{fig:tib_sensitivity_plot}
\end{figure}


%%% Local Variables:
%%% mode: latex
%%% TeX-master: "../Jensen_Shape_Sensitivity"
%%% End:

\subsection{Discussion}
The findings correspond closely with initial expectations regarding the sensitivity measurement of projected shapes relative to 3D object orientation and are consistent with areas challenging for JTML optimization.
Specifically, the humeral implant showed a generally smooth and minimal shape sensitivity profile, particularly for $\delta_{y}$ rotations (\cref{tab:ss-vals}).
Along this axis, the humeral implant is the most cylindrical, meaning we would not expect to see a significant change in the shape descriptor with minor $\delta_{y}$ rotations.
Furthermore, it is noteworthy that this axis presented the most significant difficulties in JTML optimization.

Similar intuitive outcomes are observed with the glenosphere implant, which exhibited the lowest average $\mathbb{S}(\delta)$ among all the implant types.
This implant primarily consists of an articulation surface closely approximating a spherical shape.
Given that the projection of a sphere (a circle) remains constant regardless of the sphere's orientation, the closer a shape is to a spherical form, the lower its overall shape sensitivity is expected to be.

The observed shape sensitivity of the tibial implant with respect to $\delta_{y}$ rotation aligns with the concept of symmetry traps.
There is a consistently low shape sensitivity along the line where $x=0$.
This axis, associated with internal/external rotation, is the same one that contributes to symmetry traps, where two different 3D orientations result in an identical projected shape.
In terms of this analysis, the $\Delta S$ value would be $0$ for these two orientations of the tibial implant.

This study sheds light on an important aspect of Joint Track Machine Learning, particularly the use of Euler angles in the DIRECT-JTA optimization routine.
Currently, the optimization does not involve independently varying all angles within a body-centered reference frame, as this approach is not conducive to hyperbox creation \cite{floodAutomatedRegistration3D2018,jonesLipschitzianOptimizationLipschitz1993}.
Instead, optimization is performed over a range of ordered rotations, projected through the sequence $R_{z}R_{x}R_{y}$.
The challenges the humeral implant encounters in aligning the $y$-axis illustrate that this ordered sequence, especially with a symmetric final axis, can hinder the convergence process.

Beyond the inherent shape sensitivities, such optimization limitations motivate exploring alternatives to Euler angles.
Performing registration optimization directly on the Special Orthogonal group $SO(3)$ poses an intriguing direction.
$SO(3)$ encapsulates all possible 3D rotations in a mathematically convenient structure (A \emph{Lie Group}, which is both a manifold and a group) \cite{zillerLieGroupsRepresentation2010,serreLieAlgebrasLie1992}.
By optimizing on this manifold instead of Euler angle parametrizations, issues with gimbal lock and cascading rotation effects can be avoided.
However, most manifold optimization is specifically gear toward derivative-based optimization, which is not the currently supported by the DIRECT-JTA algorithm that Joint Track Machine Learning incorporates \cite{jensenJointTrackMachine2023,jonesLipschitzianOptimizationLipschitz1993,floodAutomatedRegistration3D2018}.
This limitation motivates either a restructuring of the current optimization methods to incorporate derivative information, or to explore the potentials of derivative-free (black-box) optimization \cite{audetDerivativeFreeBlackboxOptimization2017} over manifolds.


Historical manual registration software heavily relied on bony landmarks within images to disentangle challenging poses.
As landmarks, the tibial tuberosity, fibula, and bicepital groove were vital in identifying the specific orientation of implants.
To accurately assess implant kinematics in highly symmetric unicompartmental knee arthroplasty, surface meshes of femoral and tibial components were incorporated into the registration process \cite{banksComparingVivoKinematics2005}.
These bone models were particularly beneficial in addressing implants' internal/external rotation along semi-symmetric axes.
Additionally, the use of densely selected keypoints in the precise measurement of preoperative shoulder kinematics through biplane fluoroscopy has been documented \cite{burtonFullyAutomaticTracking2023}, with keypoint registration achieved via a modified Perspective-N-Points optimization routine.
Furthermore, intensity-based metrics have played a crucial role in the accurate determination of non-operative tibiofemoral kinematics \cite{bakaStatisticalShapeModelBased2012}.
Rather than minimizing the Euclidean distance of projected model keypoints, these methods leverage digitally reconstructed radiographs and directly compare the intensity values of the x-ray image and the projection estimate.
Thus, utilizing bony landmarks for measuring implant kinematics presents a robust and reliable approach to measuring accurate joint kinematics in clinical environments.



%%% Local Variables:
%%% mode: latex
%%% End:

\section{Concluding Remarks}
This study demonstrates intrinsic differences between implant types regarding projected 2D shape sensitivity.
Measurement difficulties aligned with low sensitivity along problematic axes—humeral internal rotation and tibial symmetry traps.
Fundamentally, small orientation changes yielded negligible 2D variability for near-symmetrical geometries and axes.
While inherent shape constraints limit data extractable solely from single-plane fluoroscopic silhouettes, incorporating additional image information like bone offers promise.
Despite unavoidable ambiguity along select dimensions, boosting descriptor sensitivity and employing precise anatomical constraints could enable robust clinical tracking.
Overall, relating optimization performance to shape response underscores routes toward accurate autonomous kinematic analysis.


%%% Local Variables:
%%% mode: latex
%%% TeX-master: "../../../Andrew_Jensen_Dissertation"
%%% End:




%%% Local Variables:
%%% mode: latex
%%% TeX-master: "../../Andrew_Jensen_Dissertation"
%%% End:



%%% Local Variables:
%%% mode: latex
%%% TeX-master: "../../Andrew_Jensen_Dissertation"
%%% End:
