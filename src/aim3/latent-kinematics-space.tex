The concept of latent space in kinematics is analogous to the latent spaces found in other domains, such as natural language processing (NLP).
In NLP, latent space refers to the underlying, hidden structure in language data that algorithms attempt to uncover \cite{jurafskySpeechLanguageProcessing2009}.
This is where the power of transformers \cite{vaswaniAttentionAllYou2017} in NLP becomes relevant to our study of kinematics.

In kinematics, each movement pattern can be thought of as a unique ``expression'' of the underlying musculoskeletal system, akin to how sentences are expressions of underlying linguistic rules and meanings in language.
By exploring the latent kinematics space, we aim to reduce the number of movements required for a comprehensive kinematics evaluation without losing the predictive power or statistical significance of the evaluation.
This approach is based on the hypothesis that certain movement patterns contain redundant information, which could be inferred from other, more distinct movements.
For instance, we might predict the kinematics of a ``stair rise'' from the ``walking'' pattern, assuming that the essential biomechanical information is embedded within the walking movement.
sing Emacs, you could try the following regex replace
To achieve this, we explore the use of signal processing techniques that could ``translate'' kinematics from one movement to another.
This technique is somewhat similar to how language translation models work in NLP.
The goal is to start with a maximal set of kinematics measurements for a wide array of movements, and remove those that can be inferred from the rest.

\subsection{Brief Mathematical Preliminary and Notation}
Let $\mathcal{X}_{mvt}$ be the ambient kinematics space from which a particular sample is measured, where $mvt \in \{\text{stair},\text{walk}, \text{lunge}, \cdots\}$.
Essentially, this is the space of all possible kinematics patterns for a particular movement.
For simplicity, we can assume that this space $\mathcal{X}_{mvt}$ follows some distribution, $\mu_{mvt}$.
The ``latent space'' of our kinematics is $\mathcal{Z}$ and follows some fixed distribution, $\xi$.
For a universal translator, wherein the expressiveness of outputs resembles the underlying musculoskeletal system of that particular patient, you would not have a latent space for each movement, ($\mathcal{Z}_{mvt}$), but rather, the single latent space would (hopefully) encode all relevant information about the patient.

Mathematically, we can view this translator as a collection of an encoder, $F$, which takes us from the ambient space to the latent space, $F: \mathcal{X}_{mvt,1} \mapsto \mathcal{Z}$, and a decoder, $G$ which takes us from the latent space back to our ambient space, $G: \mathcal{Z} \mapsto \mathcal{X}_{mvt,2}$. The encoder-decoder architecture is a design decision outside the scope of this dissertation, but common choices include Variational Autoencoder (VAE) and Transformer-based networks \cite{vaswaniAttentionAllYou2017}.
However, we can generally say that our encoder architecture is parametrized by some collection of variables, $\phi$, and our decoder parametrized by $\theta$.
And so, notationally, we can view this problem as minimizing the distance between $\xi_{\phi} = F_{\phi}\mu_{mvt}$ and $\xi$, as well as minimizing the distance between $\mu_{mvt,\theta} = G_{\theta}\xi_{\phi}$.
The specifics of the math are outside the scope of this dissertation, but for interested readers, I recommend reading Jong Le's Geometry of Deep Learning, Chapter 13 \cite{yeGeometryDeepLearning2022}, which discusses metrics on probability spaces, multiple forms of divergence, and the requisite geometric and statistical intuition behind how these types of architectures work.

\subsection{A Kinematics Translator}
The main goal behind a kinematics translator is to build up a rich enough representation of the latent space of a patients kinematics, $\mathcal{Z}$, with the smallest possible number of images required.
A ``translator'' achieves this by being able to determine, for example, the ``stair rise'' kinematics from the ``walking'' kinematics, without ever needing to actually capture a patient walking up the stairs.
This has obvious benefits if one is trying to reduce the clinical footprint of autonomous kinematics measurements.

\subsubsection{Methods and Data}
The training data used to build this network would come from the same published studies used for training our segmentation neural network and validating our autonomous algorithm (\Cref{sec:jtml}).
To start, kinematics were grouped by movement (e.g. walking, stair rise, lunge, etc.), and then we created ``translation pairs'' for different movements from each patients.
We standardized all kinematics data using B-splines and interpolation so that each movement was represented with 100 points.
These 100 points at each anatomic rotation/translation measurement can be thought of as a ``word'' in our translation paradigm, meaning that for any given translation task, we are translating six ``words'' into six ``words''.
Each of these 6 kinematics measurements for movement one would be the input to our network, and the desired output are the six kinematics measurements from movement two.
With a sufficient amount of data, this would build up a rich representation of our latent space, $\mathcal{Z}$, such that it can be directly sampled or analyzed to learn about the underlying kinematics of a patient.
Unfortunately, a lack of standardization caused three major problems when performing this analysis: (1) Different groups calling the different movements the same thing, (2) Different groups using different movements, and (3) Different groups measuring kinematics at different and non-standard flexion angles.

\subsection{Problem 1: Groups calling different movements the same thing}
Even with a brief literature search, one can find many examples of different groups calling the different activities the same name.
The three most pernicious movements were ``squat'', ``lunge'', and ``deep knee bend''. ``Step Up'' also had inconsistencies in the height of the stair.
Typically, research done by the same group was standardized, but between groups there was minor resemblance, but never quite the same exact movement.

A lunge ranged between both feet on the floor, to a raised stair of variable height. The distance between the feet was typically not specified.
When images were shown, the width of the feet ranged from nearly back-toe touching front-heel, to a more standard lunge with at least a foot between the back foot and front foot.

A squat was t
\subsection{Problem 2: No standardized set of movements to measure}
\subsection{Problem 3: Groups measuring kinematics at different flexion angles}

%%% Local Variables:
%%% mode: latex
%%% TeX-master: "../../Jensen-Lit-Review"
%%% End:
