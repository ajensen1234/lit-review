The concept of latent space in kinematics is analogous to the latent spaces found in other domains, such as natural language processing (NLP).
In NLP, latent space refers to the underlying, hidden structure in language data that algorithms attempt to uncover \cite{jurafskySpeechLanguageProcessing2009}.
This is where the power of transformers \cite{vaswaniAttentionAllYou2017} in NLP becomes relevant to our study of kinematics.

In kinematics, each movement pattern can be thought of as a unique ``expression'' of the underlying musculoskeletal system, akin to how sentences are expressions of underlying linguistic rules and meanings in language.
By exploring the latent kinematics space, we aim to reduce the number of movements required for a comprehensive kinematics evaluation without losing the predictive power or statistical significance of the evaluation.
This approach is based on the hypothesis that certain movement patterns contain redundant information, which could be inferred from other, more distinct movements.
For instance, we might predict the kinematics of a ``stair rise'' from the ``walking'' pattern, assuming that the essential biomechanical information is embedded within the walking movement.

Test

%%% Local Variables:
%%% mode: latex
%%% TeX-master: "../../Jensen-Lit-Review"
%%% End:
