The concept of latent space in kinematics is analogous to the latent spaces found in other domains, such as natural language processing (NLP).
In NLP, latent space refers to the underlying, hidden structure in language data that algorithms attempt to uncover \cite{jurafskySpeechLanguageProcessing2009}.
This is where the power of transformers \cite{vaswaniAttentionAllYou2017} in NLP becomes relevant to our study of kinematics.

In kinematics, each movement pattern can be thought of as a unique ``expression'' of the underlying musculoskeletal system, akin to how sentences are expressions of underlying linguistic rules and meanings in language.
By exploring the latent kinematics space, we aim to reduce the number of movements required for a comprehensive kinematics evaluation without losing the predictive power or statistical significance of the evaluation.
This approach is based on the hypothesis that certain movement patterns contain redundant information, which could be inferred from other, more distinct movements.
For instance, we might predict the kinematics of a ``stair rise'' from the ``walking'' pattern, assuming that the essential biomechanical information is embedded within the walking movement.
sing Emacs, you could try the following regex replace
To achieve this, we explore the use of signal processing techniques that could ``translate'' kinematics from one movement to another.
This technique is somewhat similar to how language translation models work in NLP.
The goal is to start with a maximal set of kinematics measurements for a wide array of movements, and remove those that can be inferred from the rest.

\subsection{Brief Mathematical Preliminary and Notation}
Let $\mathcal{X}_{mvt}$ be the ambient kinematics space from which a particular sample is measured, where $mvt \in \{\text{stair},\text{walk}, \text{lunge}, \cdots\}$.
Essentially, this is the space of all possible kinematics patterns for a particular movement.
For simplicity, we can assume that this space $\mathcal{X}_{mvt}$ follows some distribution, $\mu_{mvt}$.
The ``latent space'' of our kinematics is $\mathcal{Z}$ and follows some fixed distribution, $\xi$.
For a universal translator, wherein the expressiveness of outputs resembles the underlying musculoskeletal system of that particular patient, you would not have a latent space for each movement, ($\mathcal{Z}_{mvt}$), but rather, the single latent space would (hopefully) encode all relevant information about the patient.

Mathematically, we can view some translator function, $T$, typically recommended as some neural network with parameters $\Theta$.

%%% Local Variables:
%%% mode: latex
%%% TeX-master: "../../Jensen-Lit-Review"
%%% End:
