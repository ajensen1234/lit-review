\chapter{Musings on Latent Kinematics Space and Synthetic Biomechanics Data}

In order to make autonomously measuring joint kinematics a clinically viable tool, the movements and motions being measured must be standardized.
Similar efforts have been put into standardized post-operative outcome metrics like the Knee Society Score (KSS)\cite{insallRationaleKneeSociety1989}, Knee Injury and Osteoarthritis Outcome Score (KOOS) \cite{roosKneeInjuryOsteoarthritis2003}, the Forgotten Joint Score \cite{behrendForgottenJointUltimate2012}, the Western Ontario and McMaster Universities Osteoarthritis Index (WOMAC) \cite{bellamy1988validation}, among others.
Standardizing these metrics enables reliable and objective comparisons across different groups, implants, and surgical techniques.
This same need for standardized metrics exists when quantifying joint kinematics for research and clinical purposes.
Unfortunately, no such standardization exists for quantifying joint kinematics.
While most groups perform similar activities of daily living (Some form of walking, stair rise, chair rise, kneel, lunge, and squat), variations in how these activities are executed hinder direct comparison accuracy.

Thus, this chapter describes a data-driven approach toward being able to select the highest yield activities to be included in a clinical kinematics examination.
To render the autonomous measurement of joint kinematics a viable clinical tool, the time to collect the data must not be excessive, and so a brief and succinct set of activities is preferred.
From these constraints, the first approach stems from the thought, ``All kinematics data is coming from the same patient. Are there underlying characteristics from one activity that translate to all activities?''
An example might arise: ``Can we predict the patient's stair rise kinematics from their walking kinematics?''
The motivating intuition behind these ideas is that they all come from the same patient, and are being driven by the same underlying musculature.
The following chapter will detail the beginning of an experimental procedure, and why the data was insufficient for answering the above questions.
However, another question arises: ``Given the statistical parameters surrounding a ``good outcome'' or ``bad outcome'', might we be able to generated synthetic patient data to overcome the lack of real-world kinematics data?''
This question is particularly relevant in the context of 'big data' and its growing impact.
Many modern algorithms leverage an enormous amount of data, often more than traditional methods of quantified biomechanics are able to generate.
Additionally, HIPAA requirements make it incredibly difficult to share data, slowing down the rapid progress in curated algorithms for biomechanics data that we see with open datasets like ImageNet \cite{russakovskyImageNetLargeScale2015}.

The following chapter will contain some thoughts on solving these problems.

\section{Latent Kinematics Space and Reducing the Number of Movements Required for a Full Kinematics Evaluation}
\label{sec:latent-kinematics-space}
The concept of latent space in kinematics is analogous to the latent spaces found in other domains, such as natural language processing (NLP).
In NLP, latent space refers to the underlying, hidden structure in language data that algorithms attempt to uncover \cite{jurafskySpeechLanguageProcessing2009}.
This is where the power of transformers \cite{vaswaniAttentionAllYou2017} in NLP becomes relevant to our study of kinematics.

In kinematics, each movement pattern can be thought of as a unique ``expression'' of the underlying musculoskeletal system, akin to how sentences are expressions of underlying linguistic rules and meanings in language.
By exploring the latent kinematics space, we aim to reduce the number of movements required for a comprehensive kinematics evaluation without losing the predictive power or statistical significance of the evaluation.
This approach is based on the hypothesis that certain movement patterns contain redundant information, which could be inferred from other, more distinct movements.
For instance, we might predict the kinematics of a ``stair rise'' from the ``walking'' pattern, assuming that the essential biomechanical information is embedded within the walking movement.

Test

%%% Local Variables:
%%% mode: latex
%%% TeX-master: "../../Jensen-Lit-Review"
%%% End:



\section{Toward Generating Synthetic Biomechanics Data}
\label{sec:synthetic-biomechanics-data}
Many of the issues encountered during the ``kinematics translator'' are directly applicable to generating synthetic kinematics data as well.
At the core, the fundamental problem is the same: we want to convert an interpretable latent space into some kinematics measurement that can be used for neural network training for kinematics analysis.
However, this still requires a robust latent space from which we draw from to generate the synthetic data, which runs into exactly the same difficulties as trying to build a kinematics translator.
The lack of standardization of kinematics measurements, as well as consistently reported post-operative outcomes, means that it would be nearly impossible to train a latent space that is both interpretable (e.g. being able to generate a ``healthy'' or ``failing'' kinematics sequence) and robust (i.e. $\xi_{\phi} \pm \delta = \xi$).

One of the main benefits of having a tool to easily, accurately, and autonomously measure kinematics data is that the door will be open to creating standard procedures for these measurements.
The next steps toward utilizing this technology in a clinical setting will be establishing a framework for all clinicians and researchers to use when performing a kinematics evaluation.
With the spread of anonymized health datasets, a kinematics dataset containing thousands of patients all performing the same activities, and each having the same post-operative metrics recorded, would pave the way for quantitative assessment and correlation of kinematics data to outcome.

%%% Local Variables:
%%% mode: latex
%%% TeX-master: "../../Jensen-Lit-Review"
%%% End:





%%% Local Variables:
%%% mode: latex
%%% TeX-master: "../../Andrew_Jensen_Dissertation"
%%% End:
