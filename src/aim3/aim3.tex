\chapter{Musings on Latent Kinematics Space and Synthetic Biomechanics Data}

\section{Introduction}
In order to make autonomously measuring joint kinematics a clinically viable tool, the movements and motions being measured must be standardized.
Similar efforts have been put into standardized post-operative outcome metrics like the Knee Society Score (KSS), Knee Injury and Osteoarthritis Outcome Score (KOOS), the Forgotten Joint Score, the Western Ontario and McMaster Universities Osteoarthritis Index (WOMAC), among others.
Standardizing these metrics enables reliable and objective comparisons across different groups, implants, and surgical techniques.
This same need for standardized metrics exists when quantifying joint kinematics for research and clinical purposes.
Unfortunately, no such standardization exists for quantifying joint kinematics.
While most groups perform similar activities of daily living (Some form of walking, stair rise, chair rise, kneel, lunge, and squat), variations in how these activities are executed hinder direct comparison accuracy.

So, this chapter describes a data-driven approach toward being able to select the highest yield activities to be included in a clinical kinematics examination.
o render the autonomous measurement of joint kinematics a viable clinical tool, the time to collect the data must not be excessive, and so a brief and succinct set of activities is preferred.
From these constraints, the first approach stems from the thought, ``All kinematics data is coming from the same patient. Are there underlying characteristics from one activity that translate to all activities?''
An example might arise: ``Can we predict the patient's stair rise kinematics from their walking kinematics?''
The motivating intuition behind these ideas is that they all come from the same patient, and are being driven by the same underlying musculature.
The following chapter will detail the beginning of an experimental procedure, and why the data was insufficient for answering the above questions.
However, another question arises: ``Given the statistical parameters surrounding a ``good outcome'' or ``bad outcome'', might we be able to generated synthetic patient data to overcome the lack of real-world kinematics data?''
This question is particularly relevant in the context of 'big data' and its growing impact.
Many modern algorithms leverage an enormous amount of data, often more than traditional methods of quantified biomechanics are able to generate.
Additionally, HIPAA requirements make it incredibly difficult to share data, slowing down the rapid progress in curated algorithms for biomechanics data that we see with open datasets like ImageNet \cite{russakovskyImageNetLargeScale2015}.

And so, the following chapter will contain some background information toward solving these problems.

\section{Latent Kinematics Space and Reducing the Number of Movements Required for a Full Kinematics Evaluation}
\label{sec:latent-kinematics-space}
The concept of latent space in kinematics is analogous to the latent spaces found in other domains, such as natural language processing (NLP).
In NLP, latent space refers to the underlying, hidden structure in language data that algorithms attempt to uncover \cite{jurafskySpeechLanguageProcessing2009}.
This is where the power of transformers \cite{vaswaniAttentionAllYou2017} in NLP becomes relevant to our study of kinematics.

In kinematics, each movement pattern can be thought of as a unique ``expression'' of the underlying musculoskeletal system, akin to how sentences are expressions of underlying linguistic rules and meanings in language.
By exploring the latent kinematics space, we aim to reduce the number of movements required for a comprehensive kinematics evaluation without losing the predictive power or statistical significance of the evaluation.
This approach is based on the hypothesis that certain movement patterns contain redundant information, which could be inferred from other, more distinct movements.
For instance, we might predict the kinematics of a ``stair rise'' from the ``walking'' pattern, assuming that the essential biomechanical information is embedded within the walking movement.
sing Emacs, you could try the following regex replace
To achieve this, we explore the use of signal processing techniques that could ``translate'' kinematics from one movement to another.
This technique is somewhat similar to how language translation models work in NLP.
The goal is to start with a maximal set of kinematics measurements for a wide array of movements, and remove those that can be inferred from the rest.

\subsection{Brief Mathematical Preliminary and Notation}
Let $\mathfrak{X}_{mvt}$ be the ambient kinematics space from which a particular sample is measured, where $mvt \in \{\text{stair},\text{walk}, \text{lunge}, \cdots\}$.
Essentially, this is the space of all possible kinematics patterns for a particular movement.
For simplicity, we can assume that this space $\mathfrak{X}_{mvt}$ follows some distribution, $\mu_{mvt}$.
The ``latent space'' of our kinematics is $\mathfrak{Z}$ and follows some fixed distribution, $\xi$.
For a universal translator, wherein the expressiveness of outputs resembles the underlying musculoskeletal system of that particular patient, you would not have a latent space for each movement, ($\mathfrak{Z}_{mvt}$), but rather, the single latent space would (hopefully) encode all relevant information about the patient.

Mathematically, we can view this translator as a collection of an encoder, $F$, which takes us from the ambient space to the latent space, $F: \mathfrak{X}_{mvt,1} \mapsto \mathfrak{Z}$, and a decoder, $G$ which takes us from the latent space back to our ambient space, $G: \mathfrak{Z} \mapsto \mathfrak{X}_{mvt,2}$. The encoder-decoder architecture is a design decision outside the scope of this dissertation, but common choices include Variational Autoencoder (VAE) and Transformer-based networks \cite{vaswaniAttentionAllYou2017}.
However, we can generally say that our encoder architecture is parametrized by some collection of variables, $\phi$, and our decoder parametrized by $\theta$.
And so, notationally, we can view this problem as minimizing the distance between $\xi_{\phi} = F_{\phi}\mu_{mvt}$ and $\xi$, as well as minimizing the distance between $\mu_{mvt,\theta} = G_{\theta}\xi_{\phi}$ and $\mu$.
The specifics of the math are outside the scope of this dissertation, but for interested readers, I recommend reading Jong Le's Geometry of Deep Learning, Chapter 13 \cite{yeGeometryDeepLearning2022}, which discusses metrics on probability spaces, multiple forms of divergence, and the requisite geometric and statistical intuition behind how these types of architectures work.

\subsection{A Kinematics Translator}
The main goal behind a kinematics translator is to build up a rich enough representation of the latent space of a patients kinematics, $\mathfrak{Z}$, with the smallest possible number of images required.
A ``translator'' achieves this by being able to determine, for example, the ``stair rise'' kinematics from the ``walking'' kinematics, without ever needing to actually capture a patient walking up the stairs.
This has obvious benefits if one is trying to reduce the clinical footprint of autonomous kinematics measurements.

\subsubsection{Methods and Data}
The training data used to build this network would come from the same published studies used for training our segmentation neural network and validating our autonomous algorithm (\Cref{sec:jtml}).
To start, kinematics were grouped by movement (e.g. walking, stair rise, lunge, etc.), and then we created ``translation pairs'' for different movements from each patients.
We standardized all kinematics data using B-splines and interpolation so that each movement was represented with 100 points.
These 100 points at each anatomic rotation/translation measurement can be thought of as a ``word'' in our translation paradigm, meaning that for any given translation task, we are translating six ``words'' into six ``words''.
Each of these 6 kinematics measurements for movement one would be the input to our network, and the desired output are the six kinematics measurements from movement two.
With a sufficient amount of data, this would build up a rich representation of our latent space, $\mathfrak{Z}$, such that it can be directly sampled or analyzed to learn about the underlying kinematics of a patient.
Unfortunately, a lack of standardization caused three major problems when performing this analysis: (1) Different groups calling the different movements the same thing, (2) Different groups using different movements, and (3) Different groups measuring kinematics at different and non-standard flexion angles.

\subsection{Problem 1: Groups calling different movements the same thing}
Even with a brief literature search, one can find many examples of different groups calling the different activities the same name.
The three most pernicious movements were ``squat'', ``lunge'', and ``deep knee bend''. ``Step Up'' also had inconsistencies in the height of the stair.
Typically, research done by the same group was standardized, but between groups there was minor resemblance, but never quite the same exact movement.

A lunge ranged between both feet on the floor, to a raised stair of variable height. The distance between the feet was typically not specified.
When images were shown, the width of the feet ranged from nearly back-toe touching front-heel, to a more standard lunge with at least a foot between the back foot and front foot.

A squat typically involves both feet next to each other, and the participants bending down as far as they can.
However, the placement of the feet with respect to each other (narrow or wide), and the direction of the toes (straight forward or pointed outward) were often not specified.
Additionally, some groups defined a squat as having staggered feet, which bears striking similarity to what other groups appear to call a ``deep knee bend''.
Unfortunately, the specifications of the movements were often not provided, just the name, causing tremendous difficulty is pairing different kinematics sequences between research groups.
\subsection{Problem 2: No standardized set of movements to measure}
Because kinematics measurements are primarily a research tool and have not been adopted as a clinical standard of care, the activities performed are entirely at the discretion of the research group.
This leads to different research groups having their study participants perform different sets of activities.
When trying to build a universal translator, one of the key components is having a large enough set of paired movements from single patients.
For example, a large set of both ``walking'' and ``stair rise'' kinematics are needed to reliably build a network that could translate between the two.
Unfortunately, few groups measured the same two movements reliably, leading to a paucity of paired kinematics data on which a translation algorithm could be built.

\subsection{Problem 3: Groups measuring kinematics at different flexion angles}
Another key issue in the proposed datasets is the flexion-angle resolution at which different groups measure kinematics.
Some groups measure kinematics at every $n^{th}$ frame, typically resulting in a flexion resolution of $2-5^{\circ}$.
These measurements are able to accurately capture the knee during the entire range of motion, and hopefully, provide a robust analysis of small kinematics changes between patients.

However, other groups only measure kinematics at prescribed flexion angles (typically $30,60,90,120$), which loses resolution and comparative power for small motions.
If, for example, the main difference between a healthy and pathological post-operative knee occurs between $30-60^{\circ}$, then these lower resolution measurements will be relatively useless at providing meaningful data for a comprehensive database or kinematics translation paradigm.
The interpolated values between these two extrema of measurement would fill the latent space with, at best, useless, at worst, completely
incorrect, values.

Of note, one of the main reasons for the choice of performing measurements at these prescribed angles, which typically only requires $3/4$ measurements per movement sequence, is the time to generate a report.
Manual registration is time-consuming, and some groups do not feel that the extra time to measure the entirety of the movement is worth the additional accuracy.
However, with widespread adoption of the proposed Joint Track Machine Learning (\Cref{sec:jtml}), the human supervision for these measurements is completely eliminated, allowing these groups to take more robust measurements without the previously necessary manual registration.



%%% Local Variables:
%%% mode: latex
%%% TeX-master: "../../Andrew_Jensen_Dissertation"
%%% End:



\section{Toward Generating Synthetic Biomechanics Data}
\label{sec:synthetic-biomechanics-data}
Many of the issues encountered during the ``kinematics translator'' are directly applicable to generating synthetic kinematics data as well.
At the core, the fundamental problem is the same: we want to convert an interpretable latent space into some kinematics measurement that can be used for neural network training for kinematics analysis.
However, this still requires a robust latent space from which we draw from to generate the synthetic data, which runs into exactly the same difficulties as trying to build a kinematics translator.
The lack of standardization of kinematics measurements, as well as consistently reported post-operative outcomes, means that it would be nearly impossible to train a latent space that is both interpretable (e.g. being able to generate a ``healthy'' or ``failing'' kinematics sequence) and robust (i.e. $\xi_{\phi} \pm \delta = \xi$).

One of the main benefits of having a tool to easily, accurately, and autonomously measure kinematics data is that the door will be open to creating standard procedures for these measurements.
The next steps toward utilizing this technology in a clinical setting will be establishing a framework for all clinicians and researchers to use when performing a kinematics evaluation.
With the spread of anonymized health datasets, a kinematics dataset containing thousands of patients all performing the same activities, and each having the same post-operative metrics recorded, would pave the way for quantitative assessment and correlation of kinematics data to outcome.





%%% Local Variables:
%%% mode: latex
%%% TeX-master: "../../Andrew_Jensen_Dissertation"
%%% End:
