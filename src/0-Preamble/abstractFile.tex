The fundamental role of human synovial joints is enabling dynamic motion within the musculoskeletal system, which is essential for a wide range of activities.
Diseases affecting these joints often lead to pain during movement, imposing significant economic burdens estimated at approximately USD 300 billion annually, encompassing direct healthcare costs and indirect impacts such as lost productivity \cite{BMUSBurdenMusculoskeletal}.
Among these conditions, osteoarthritis is particularly prevalent, characterized by the gradual deterioration of cartilage, leading to joint pain and mobility issues \cite{sharmaOsteoarthritisCompanionRheumatology2007}.
Arthroplasty, specifically total knee arthroplasty (TKA) and reverse total shoulder arthroplasty (rTSA), presents a solution by replacing the damaged joint surfaces with synthetic materials to alleviate pain and restore functionality.
Despite the high success rate, a subset of patients reports dissatisfaction due to residual symptoms such as pain, instability, or stiffness post-surgery \cite{bakerRolePainFunction2007,scottPredictingDissatisfactionFollowing2010, bournePatientSatisfactionTotal2010}.
Current diagnostic practices for evaluating musculoskeletal disorders primarily rely on static assessments, which fail to capture the complexities of joint mechanics under dynamic conditions—when symptoms are most pronounced.
While sophisticated technologies for 3D motion capture exist, their adoption in clinical settings is limited by high costs and operational challenges. Combined with 3D-to-2D model-image registration, fluoroscopic imaging has shown promise for accurate joint assessment.
However, the technique's broader application is hampered by the extensive manual effort required, leading to increased time and financial resources \cite{banksAccurateMeasurementThreedimensional1996, mahfouzRobustMethodRegistration2003}.
This dissertation introduces a novel framework for the autonomous measurement of TKA and rTSA kinematics utilizing single-plane imaging techniques.
It integrates both modern and historical algorithms, leveraging recent advancements in computational power and machine learning to eliminate the need for manual intervention in the model-image registration process.
This approach promises a reliable, cost-effective, and practical method for obtaining accurate dynamic joint kinematics in a clinical setting, potentially transforming the diagnostic landscape for musculoskeletal diseases and enhancing patient outcomes \cite{postolkaEvaluationIntensitybasedAlgorithm2020,floodAutomatedRegistration3D2018}.

%%% Local Variables:
%%% mode: latex
%%% TeX-master: "../../Andrew_Jensen_Dissertation"
%%% End:
