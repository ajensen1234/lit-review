3D/2D model-image registration leverages computer vision to determine position and orientation of a 3D model within a 2D image.
The key premise is that the real-world camera's projective properties can be computationally emulated.
An image similarity metric can then quantify alignment between the projected 3D model (\cref{eq:perspective-projection}, \cref{sec:img-form-camera-props}) and the 2D image (\cref{sec:image-similarity}).
Optimization seeks transformations minimizing this metric, thereby ``placing'' the model accurately.
Various techniques have been explored for registering total knee arthroplasty components, exhibiting tradeoffs in computational complexity, choice of similarity metric, and optimization methodology.

\subsection{Pre-Computed Projective Geometries}
Early registration methods were constrained by limited computational resources, necessitating efficient alignments without iterative projections.
Established techniques utilized pre-computed descriptors enabling rapid pose optimization.


First established in the early 1970s, Normalized Fourier Descriptors provide rotationally invariant image representations derived from 2D shape information \cite{wallaceAnalysisThreedimensionalMovement1980,richardIdentificationThreeDimensionalObjects1974,persoonShapeDiscriminationUsing1977,wallaceEfficientThreedimensionalAircraft1980}.
Given camera intrinsic parameters, these enable accurate TKA tracking by minimizing descriptor differences to a descriptor library across known poses \cite{banksModelBased3D1992,banksAccurateMeasurementThreedimensional1996}.

In parallel, another group utilized pre-computed distance maps intrinsic to the 3D model \cite{lavalleeRecoveringPositionOrientation1995,zuffiModelbasedMethodReconstruction1999}.
These distance maps could then be used to quickly determine the Euclidean distance between any node in the model and an arbitrary line in 3D space.
Then, 3D vectors were creating starting at sampled points along the contour of the implant, and concluding at the origin of the camera.
Assuming an accurate focal distance, then the objective is simply minimizing the distance between all the vectors and their corresponding nearest node on the 3D model.
Once minimized, the 3D object fell into the ``slot'' carved out for it by the pseudo-conical shape.


However, both approaches rely on manual segmentation of relevant implant contours.
Despite automated edge detectors \cite{cannyComputationalApproachEdge1986}, manual selection of salient edges remains cumbersome and clinically impractical.


\subsection{Motion Capture}

Motion capture refers to the tracking of reflective markers using multiple cameras to overdetermine marker locations. If marker positions relative to anatomical features are known, transformation matrices can sequentially determine anatomical kinematics from observable marker trajectories.

Historically, two approaches have been explored - skin-mounted and bone-mounted motion tracking.
Skin-mounted markers utilize fiducial points proximate to joint centers, enabling approximation of translations and rotations.
However, numerous studies have demonstrated substantial skin motion during activity, significantly reducing joint measurement accuracy \cite{gaoInvestigationSoftTissue2008,garlingSofttissueArtefactAssessment2007,linEffectsSoftTissue2016,kuoInfluenceSoftTissue2011}.
Such uncertainties render this methodology insufficiently accurate for clinical joint kinematic quantification.

Alternatively, bone-drilled markers and pins enable direct skeletal tracking with greater precision \cite{lafortuneThreedimensionalKinematicsHuman1992}.
Although highly accurate, the required surgical procedures are prohibitively invasive and time-intensive for widespread clinical adoption.

\subsection{Iterative Projections}
Improved computational capabilities have enabled registration techniques reliant upon iterative projections previously hindered by complexity.
Parallel computing, in particular, delivered substantial accelerations to computer graphics performance.

By leveraging real-time rendering of projected 3D models, direct image similarity metrics avoid pre-computation in favor of online alignment optimization.
Groups have achieved robust results applying simulated annealing with image matching \cite{mahfouzRobustMethodRegistration2003}, and Lipschitzian optimization on contour correspondences \cite{floodAutomatedRegistration3D2018}.
However, escaping local minima and setting initial pose estimates still require human guidance.
Performance also remains sensitive to image noise and edge detection hyperparameters, limiting generalization.


\subsection{Fully Human Supervised}

JointTrack software exemplifies a highly accurate system combining human anatomical expertise with computational projection emulation \cite{muJointTrackOpenSourceEasily2007}.
Users manipulate 3D bone models to align with 2D fluoroscopy, leveraging intuitive human visual pattern recognition and motor control.
Additional tooling, like coronal viewing or kinematic graphs, facilitates rapid identification and correction of tracking errors.

Hundreds of papers have been published using this software as the method for determining kinematics, and it has been extensively validated using many different methods offering ground-truth solutions to kinematics.
The primary challenges with this method include the substantial time required for user training and the need for continuous human supervision during the measurement process.

\subsection{Biplane Kinematics Measurements}

Biplane fluoroscopic systems add a second camera, enhancing 3D positioning accuracy by resolving single-view ambiguities \cite{burtonAutomaticTrackingHealthy2021,youVivoMeasurement3D2001,bakaStatisticalShapeModelBased2012}.
Out-of-plane translations become tractable by exploiting cross-camera invariance.
Sub-millimeter, sub-degree precision has been reported across all dimensions.

While this seems extremely promising, the general cost and unavailability of bi-plane fluoroscopic imaging systems presents a problem for integrating this technology into a clinical setting.
For kinematic analysis to achieve widespread clinical adoption, it must be seamlessly integrated into the imaging systems present at most hospitals and clinics, which is single-plane systems.
However, the accuracy of bi-plane systems can be used to validation of performance for various single-plane pipelines \cite{brobergValidationMachineLearning2023}.

\subsection{Roentgen Stereophotogrammetric Analysis}

Roentgen stereophotogrammetry, introduced in the 1970s, achieves exceptional accuracy by tracking high-contrast tantalum beads implanted into structures of interest \cite{selvikRoentgenStereophotogrammetryMethod1989}.
Most often, it is used to determine implant micromotion in post-operative followups due to the extremely high resolution at which it measures pose and orientation.

The largest limitation of this method is two-fold (1) the need for additional surgical steps precludes this from being a widely applied method, and (2) the need for biplane imaging system poses a problem to availability and portability.


%%% Local Variables:
%%% mode: latex
%%% TeX-master: "../../../Andrew_Jensen_Dissertation"
%%% End:
