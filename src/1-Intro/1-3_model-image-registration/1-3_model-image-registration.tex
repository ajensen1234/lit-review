3D/2D model-image registration utilizes the computer vision principles discussed previously to determine the position and orientation of the model, given an image containing that model. Relying on the idea that the projective scheme of the real-world camera can be emulated, then the goal is to find some type of similarity metric (\cref{sec:image-similarity}) between a projected 3D model (\cref{eq:perspective-projection}, \cref{sec:img-form-camera-props}) and the existing image such that minimizing this metric indicates that our object has been ``placed'' correctly. Many different branches of computer vision and optimization have been explored for determining the kinematics of total knee arthroplasty components, each with varying levels of computational intensity, objective function (via image similarity), and optimization routine. General groupings of each will be discussed here, along with pitfalls and limitations that prevent the technique from being applicable in a clinical setting.