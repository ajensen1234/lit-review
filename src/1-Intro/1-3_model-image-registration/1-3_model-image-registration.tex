3D/2D model-image registration utilizes the computer vision principles discussed previously to determine the position and orientation of the model, given an image containing that model. Relying on the idea that the projective scheme of the real-world camera can be emulated, then the goal is to find some type of similarity metric (\cref{sec:image-similarity}) between a projected 3D model (\cref{eq:perspective-projection}, \cref{sec:img-form-camera-props}) and the existing image such that minimizing this metric indicates that our object has been ``placed'' correctly. Many different branches of computer vision and optimization have been explored for determining the kinematics of total knee arthroplasty components, each with varying levels of computational intensity, objective function (via image similarity), and optimization routine. General groupings of each will be discussed here, along with pitfalls and limitations that prevent the technique from being applicable in a clinical setting.

\subsection{Pre-Computed Projective Geometries}
The earliest methods of model-image registration had to deal with many of the limiting factors of computational availability at the time. This forced researchers to find clever ways to determine image similarity metrics without the need to iteratively compute the projective geometry of the model thousands of times per second.

First established in the early 80s \cite{wallaceAnalysisThreedimensionalMovement1980,wallaceEfficientThreedimensionalAircraft1980}, normalized Fourier descriptors provide a way to normalize 2-dimensional images using information from the latent 3D characteristics (position and orientation). This was used to determine TKA kinematics to high levels of accuracy \cite{banksModelBased3D1992,banksAccurateMeasurementThreedimensional1996}, so long as 3D shape information was known, and the camera matrix could be deduced. The image similarity metric utilized the $L_2$ difference between the normalized fourier descriptors of the input image and the precomputed shape library at known rotations. Further interpolation was used to increase accuracy significantly.

In parallel, another group utilized pre-computed distance maps intrinsic to the 3D model \cite{lavalleeRecoveringPositionOrientation1995,zuffiModelbasedMethodReconstruction1999}. These distance maps could then be used to quickly determine the Euclidean distance between any node in the model and an arbitrary line in 3D space. Then, 3D vectors were creating starting at sampled points along the contour of the implant, and concluding at the origin of the camera. Assuming an accurate focal distance, then the objective is simply minimizing the distance between all the vectors and their corresponding nearest node on the 3D model. Once minimized, the 3D object fell into the ``slot'' carved out for it by the psuedo-conical shape.


The main issue with these pre-computed geometries is need for the researchers to hand-select the contours belonging to the implant. Despite the recently available Canny edge-detector \cite{cannyComputationalApproachEdge1986}, one still needed to hand-label the specific edges of interest. This would be far too time consuming in a clinical setting.