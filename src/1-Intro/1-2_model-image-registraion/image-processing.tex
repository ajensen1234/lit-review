Digital image processing is a field of computer vision that deals with the manipulation, analysis, and interpretation of digital images. It focuses on algorithms and techniques that extract meaningful information from images and to enhance visual quality. In fluoroscopy, image processing can be used to enhance the digital image, or determine important information about it such that a the model-image registration task can be performed efficiently and autonomously.

\subsubsection{Filtering and Convolution}
We have already seen how image formation yields a collection of 2D points, $\mathbf{x}_{pix}$. We can write the intensity values at each pixel locations as a function, $f(\mathbf{x}_{pix}) = f(i,j)$, where $(i,j)$ represent locations in the image, and the function returns the intensity of the image at that particular pixel location. This allows us to treat images as functions, and perform similar functional operations and analysis to extract meaningful information from them.

The most widely used filter is a linear filter \cite{szeliskiComputerVisionAlgorithms2022}, where the output is some linear operation on the neighboring pixels (Eq. \ref{eq:convolution}), also known as a \emph{convolution}. In a convolution, the kernel, $h$ is shifted along the input image, $f$, and the resultant image, $g$, is the dot product of those two matrices at that specific location.

\begin{equation}
    \begin{aligned}
        g(i,j) &= \sum_{k,l}f(i-k,j-l)h(k,l) \\
        &= \sum_{k,l}f(k,l)h(i-k,j-l) \\
        &\text{Where we use the following notation}\\
        g&= f * h
    \end{aligned}
    \label{eq:convolution}
\end{equation}

The convolution operation is \emph{linear shift invariant}, which means that it obeys the superposition principle (Eq. \ref{eq:superposition}) and the shift invariance principle (Eq. \ref{eq:shift-invariance}). This is a powerful property, because it will behave the same everywhere on the input signal/image, which is useful for different types of feature extraction and filtering operations.

\begin{equation}
    h *(f + g) = h*f + h*g
    \label{eq:superposition}
\end{equation}

\begin{equation}
    g(i,j) = f(i+k,j+l) \Longleftrightarrow (h*g)(i,j) = (h*f)(i+k,j+l)
    \label{eq:shift-invariance}
\end{equation}

A common filter applied to images is the Gaussian kernel (Eq. \ref{eq:gaussian-kernal}). This kernel is shaped as a 2D discrete Gaussian, and has the effect of blurring an image and removing noise.

\begin{equation}
    \frac{1}{256}\begin{bmatrix}
        1 & 4 & 6 & 4 & 1 \\
        4 & 16 & 24 & 16 & 4\\
        6 & 24 & 36 & 24 & 6\\
        4 & 16 & 24 & 16 & 4\\
        1 & 4 & 6 & 4 & 1 \\
    \end{bmatrix}
    \label{eq:gaussian-kernal}
\end{equation}

Another is the box kernel, which averages the value of the nearest K pixels (Eq. \ref{eq:box-filter}).

\begin{equation}
    \frac{1}{K^{2}}\begin{bmatrix}
        1 & 1 & \cdots &1\\
        1 & 1 & \cdots &1 \\
        \vdots & \vdots & 1 & \vdots \\
        1 & 1 & \cdots & 1
    \end{bmatrix}
    \label{eq:box-filter}
\end{equation}

Edge filters can also be created to detect vertical (Eq. \ref{eq:vert-edge-filter}), horizontal (Eq. \ref{eq:horiz-edge-filter}), or diagonal edges (Eq. \ref{eq:diag-edge-filter}). As each of the filters moves across the feature it is designed for, that region of the output will be more highly activated than other regions, extracting out the desired components. The orientation of each of these filters can be hand-selected to find desirable attributes in images.

\begin{equation}
    \text{vertical edge filter} = \begin{bmatrix}
            0 & 1 & 0 \\
            0 & 1 & 0 \\
            0 & 1 & 0 \\
    \end{bmatrix}
    \label{eq:vert-edge-filter}
\end{equation}

\begin{equation}
    \text{horizontal edge filter} =\begin{bmatrix}
        0 & 0 & 0 \\
        1 & 1 & 1 \\
        0 & 0 & 0 \\
    \end{bmatrix}
    \label{eq:horiz-edge-filter}
\end{equation}

\begin{equation}
    \begin{aligned}
        \text{diagonal edge filters} = \begin{bmatrix}
            1 & 0 & 0 \\
            0 & 1 & 0\\
            0 & 0 & 1
        \end{bmatrix}& \text{and} & 
        \begin{bmatrix}
            0 & 0 & 1 \\
            0 & 1 & 0\\
            1 & 0 & 0
        \end{bmatrix}
    \end{aligned}
    \label{eq:diag-edge-filter}
\end{equation}

Lastly, we can use a corner filter to find corners in images (Eq. \ref{eq:corner-filter}).

\begin{equation}
    \frac{1}{4}\begin{bmatrix}
        1 & -2 & 1 \\
        -2 & 4 & -2 \\
        1 & -2 & 1 \\
    \end{bmatrix}
    \label{eq:corner-filter}
\end{equation}