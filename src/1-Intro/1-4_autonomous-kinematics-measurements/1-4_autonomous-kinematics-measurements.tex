The proposed method overcomes the various limitations of previous attempts to autonomously measure kinematics from single plane fluoroscopy. The key feature of all these limitations is that they prevent the adoption of this technique in a clinical setting, due to the labor overhead or equipment required in order to generate an accurate kinematic report. The proposed combination of methods will alleviate these requirements, and the extensibility of the software will allow this technology to be used in a clinical setting without disrupting the standard clinical workflow.

\subsection{Autonomous Implant Detection}
\label{sec:implant-detection}
Determining the location of an object in an image is a historically intractable problem. These are one of those tasks that are often relegated to the corner of ``easy for humans, extremely difficult for computers'', especially when there is very little a-priori information available. However, as discussed in \Cref{sec:deep-learning}, deep learning has paved the way for computer vision programs to be able to performs tasks that were once only possible by humans. Two convolutional neural networks were trained to segment that tibial and femoral components from the single plane fluoroscopic images. The network achitecture used was the High-Resolution Net \cite{wangDeepHighResolutionRepresentation2020}, which leverages low-level features with higher resolution parallel processing in order to better determine the spatial characteristics of the image and produce a better output. At the time of writing, this network sets the state-of-the-art standard for performance on the COCO and ImageNet datasets, demonstrating robustness for use in many different arenas.

\begin{center}
    \Large{put some pictures here of the segmentation performance of the neural network}
\end{center}

Many of the historic methods of determining kinematics were limited by the researchers ability to quickly and reliably determine the location in the implant. The contours were either hand-segmented \cite{banksAccurateMeasurementThreedimensional1996, zuffiModelbasedMethodReconstruction1999}, or a normal edge detector was used, which introduces extra tuning parameters for any given study due to variations in image quality.


\subsubsection{Neural Network Robustness}
One of the main problems with neural networks is overfitting. With millions of parameters to tune, it can be extremely easy to ``overfit'' on your training set, leading to the network's inability to perform well on any image that was not directly in the training set. When dealing with fluoroscopic images, this might look like a neural network that can perform extremely well on high-quality, high frame rate, low blur images from a hospital that has a budget to support such a machine, while failing to segment images from a machine more than a decade old. We overcame this challenge through a mixture of additional image augmentations \cite{buslaevAlbumentationsFastFlexible2020} and using a wide range of training data. The neural networks were trained on roughly 8000 images from 7 different human-supervised total knee kinematics studies spanning the last two decades. The image qualities range from extremely clear and high quality to nearly indiscernible without intense human supervision. The goal of this two-pronged approach was to have both artifical and real ``low quality'' images for the network to train on so that any hospital or researcher, regardless of the available equipment, might be able to leverage this technology in their practice. The authors hope that this approach provides equal access to this informative measurement.

\subsection{Initial Pose Estimation}
\label{sec:pose-estimation}
Hand-in-hand with implant detection is the initial pose estimate that very often needs to be input into the optimization routine. Once the contour of the implant was determined, many methods required a human operator to place the implant in the ``capture region'' of their optimizer in order for it to eventually find the correct solution. This takes human supervision to get correct, thus adding another impediment toward getting this technology into the clinic.

To determine an initial estimate, we must rely on those methods that can leverage information present in the camera projection (\Cref{sec:img-form-camera-props}) and the 2D CNN output (\Cref{sec:implant-detection}). The primary method that takes advantage of this information utilizing normalized fourier descriptor shape libraries, and extracting each of the 6 degrees-of-freedom from either the normalized coefficients or matching with the closest entry in the library \cite{wallaceAnalysisThreedimensionalMovement1980,wallaceEfficientThreedimensionalAircraft1980,banksModelBased3D1992,banksAccurateMeasurementThreedimensional1996}. The key feature that makes this method tractable for autonomous measurements is the availability of the implant pixels from the convolutional neural network.

\subsubsection{Generating Normalized Fourier Descriptors}
\label{sec:nfd}
The Fourier Transform is one that takes in a continuous and repeating function and represents it as a sum of sinusoidal signals. Because the implant contour is self intersection, we can view it as a continuous function that has period $2\pi$ radians, as it will loop back onto itself and start over. This allows us to take advantage of the Fast Fourier Transform (FFT) \cite{cochranWhatFastFourier1967}, which operates on a discrete set of points rather than a continuous function. First, the contour of the segmented implant is taken then resampled into 128 equi-spaced points. The choice of $2^n$ points allows the FFT algorithm to perform much more quickly than another choice of points. Each contour point on the image $(x, y)$ is then represented as a single complex point, $x = jy$, such that the 1D FFT algorithm can be used. If $s(n)$ represents the complex sequence of equi-spaced contour points in the implant, and $S(i)$ represents the frequncy-domain representation of those points after the FFT is applied, then we can represent these functions as shown in \cref{eq:fft}.

\begin{equation}
    \begin{aligned}
        S(i) &= \sum_{n = -\frac{N}{2} + 1}^{\frac{N}{2}} s(n)e^{-j(2\pi i n)/N} \\
        &\text{for } -\frac{N}{2} + 1 \le i \le \frac{N}{2} \\
        s(n) &= \frac{1}{N} \sum_{i = \frac{N}{2} + 1}^{\frac{N}{2}} S(i)e^{j(2\pi i n )/N}\\
        &\text{for } -\frac{N}{2} + 1 \le n \le \frac{N}{2}
    \end{aligned}
    \label{eq:fft}
\end{equation}

As discussed in \Cref{sec:image-similarity}, spatial information in images is difficult for computers to understand without explicit calculation. So, we use the properties of the FFT to normalize each of the shapes based on location, rotation, and size in order to accurately compare the segmented mask to a generated library.

\paragraph*{Normalize Position}
By the properties of the FFT, we know that $S(0)$ is the geometric centroid of the contour. Thus, we can set this to zero for all contours to give a consistent reference point that is independent of the location of the input contour. We can save this value as the ``position normalization coefficient'' to determine the $(x,y)$ location of the implant later in the process. And, because of our usage of the 1D FFT, we know that the real portion of this coefficient is the x-value and the imaginary portion is the y-value.

\begin{equation}
    \begin{aligned}
        \text{Position Coefficient} &\leftarrow S(0) \\
        S(0) &= 0
    \end{aligned}
\end{equation}

\paragraph*{Normalize Size}
Because the implant is not self-intersecting, we know that $S(1)$ is the coefficient with the largest size, and it also represent the scale of the shape of the imaplant. So, we can normalize each shape by dividing each coefficient by the overall size of the contour, shown below.

\begin{equation}
    \begin{aligned}
        \text{Size Coefficient} &\leftarrow |S(1)| \\
        S(i) &= \frac{S(i)}{|S(1)|} & \\
        & & \text{for } -\frac{N}{2} + 1 < i < \frac{N}{2}
    \end{aligned}
\end{equation}


\paragraph*{Normalize In-Plane Rotation and Contour Starting Point}

One of the most difficult parts of measuring the similarity between two contours, especially when they are composed of a set of discrete points, is that any distance measurement necessarily takes into account those discrete points in the order that they are presented. To illustrate this example, imagine two squares, each defined by the location of the corners $A_i,B_i,C_i,D_i$. If these two squares are perfectly overlapping, one might imagine that the Euclidean distance between each of the points, $(\sum_{P \in A,B,C,D} (P_1 - P_2)^2)^{\frac{1}{2}}$ would be equal to zero. This would only be true if the starting point of the contour was the same for each square (e.g. Corner $A$ was always the top left corner). If each square had $A$ starting in different locations, then even when the contours are perfectly aligned, the distance metric would be non-zero and uninformative. 

We can use properties of the FFT to normalize the starting position of the contour in each of the segmentations as well as the general orientation of the contour. We can leverage the starting point shift property of the fourier transform (\cref{eq:fft-starting-point}) and the rotation property (\cref{eq:fft-rotation}) in order to normalize both of these factors and ensure that similar shapes have the same orientation and starting point. 

\begin{equation}
    s(n-T) \xleftrightarrow[]{DFT} S(i)e^{-jiT}
    \label{eq:fft-starting-point}
\end{equation}

\begin{equation}
    s(n)e^{j\theta} \xleftrightarrow[]{DFT} S(i)e^{j\theta}
    \label{eq:fft-rotation}
\end{equation}

To normalize the starting point and rotation, we find $k$ such that $S(k)$ is the coefficient of second largest magnitude. We then apply a mixture of the starting point shift and the rotation property of the FFT to orient each contour (\cref{eq:fft-rot-norm}). We also want to find the ``rotation normalization coefficient'', which the the angle through which the contour needs to rotate to reach the normalized orientation. This is done by determining the phase of the normalization equation at $i = 0$, which controls the overall orientation of the contour. If $u$ is the phase of $S(1)$, and $v$ is the phase of $S(k)$, then we can find the normalized orientation.

\begin{equation}
    \begin{aligned}
    \text{Rotation Normalization Coefficient} &\leftarrow \frac{v - ku}{k-1}\\
    S(i)_{norm} &= S(i)e^{j\frac{(i-k)u + (1-i)v}{k-1}} & \\
    & & \text{for } - \frac{N}{2}+1 \le i \le \frac{N}{2}
    \end{aligned}
    \label{eq:fft-rot-norm}
\end{equation}

Once the in-plane rotations have been normalized, the contour has been completely normalized for $x, y,z$ translations and $z$ rotations. And, by storing these values, we are able to utilize them in determining an initial pose estimate. Then, we can use a library made up of known $x,y$ rotations, and compare the segmentation to this library to determine all 6 degrees-of-freedom.

\subsubsection{Shape Library}
A shape library is created using a flat panel projection (\Cref{sec:img-form-camera-props}) of the implant at known $x$ and $y$ rotations, while holding all positions and orientations constant, then applying the normalization protocol described above to determine the normalized coefficients of each library entry. If $s_{x,y}(n)$ is the flat-panel projection of the implant with $x$ and $y$ rotations, then we can generate a library using the following procedure, where $FFT$ is the fast fourier transform (\cref{eq:fft}) and $NFD$ is the process of normalizing the contour and extracting the relevant coefficients (\cref{sec:nfd}).

\begin{equation}
    S^{lib}_{x,y}(i) = NFD(FFT(s_{x,y}(n)))
    \label{eq:lib-generation}
\end{equation}

Once the shape library is generated for a specific implant, we can start to determine the pose estimates for each degree of freedom.

\subsubsection{Generating a Pose Estimate}
First, we compare the Euclidean distance of the normalized segmentation contour to each value of the shape library; the $x$ and $y$ rotations that minimize this function are taken as the $x$ and $y$ rotations of the implant (\cref{eq:library-min}).

\begin{equation}
    (\theta_{x,est},\theta_{y,est}) = \argmin_{x,y}(\sum_{i = -\frac{N}{2} + 1}^{\frac{N}{2}}(S^{seg}(i) - S^{lib}_{x,y}(i))^2)^{\frac{1}{2}}
    \label{eq:library-min}
\end{equation}

Then, we can determine the $z$ rotation estimate by comparing the values of the normalized $\theta$ values that were needed in \cref{eq:fft-rot-norm}. This process is shown in \cref{eq:z-rot-est}.

\begin{equation}
    \begin{aligned}
        \theta_{z,est} &= \theta^{seg} - \theta^{lib}_{x,y}\\
        \text{where } \theta^{seg,lib} &\equiv \text{Rotation Normalized Coefficient}
    \end{aligned}
\label{eq:z-rot-est}
\end{equation}

We can then use the principals of projective geometry (\cref{fig:perspective-projection}) and similar triangles to determine the out-of-plane translation of the implant, given that the library was projected at a known depth. Using similar triangles, we are able to determine that the depth is inversely proportional to the overall magnitude of the projection, captured by the ``Size Coefficient'' (\cref{eq:z-est}). We also make the assumption that we have a weak perspective projection, meaning that the out-of-plane translations are small compared to the focal length of the fluoroscopy imaging setup.

\begin{equation}
    \begin{aligned}
        \frac{M^{seg}}{f} &= \frac{h}{z_{est}}\\
        \frac{M^{lib}}{f} &= \frac{h}{z_{lib}}\\
        & \rightarrow \\
        z_{est}& = \frac{M^{lib}}{M^{seg}}z_{lib} \\
        &\text{where }\\
        M^{seg,lib} &\equiv \text{Size Coefficients} \\
        h &\equiv \text{Implant Size}
    \end{aligned}
    \label{eq:z-est}
\end{equation}

The $x$ and $y$ translations can be determined using the value of the $z$ translation estimate, along with the location of the centroid, saved as the ``Position Coefficient''. This is then refined further to account for the rotation of the implant and the distance of the implant's centroid to its origin. We can express this with a single matrix multiplication multiplied by a scale factor (\cref{eq:x-y-est}).

\begin{equation}
    \begin{pmatrix}
        x_{est} \\ y_{est}
    \end{pmatrix} = \begin{pmatrix}
        Re(S^{seg}(0)) \\ Imag(S^{seg}(0))
    \end{pmatrix} - \begin{pmatrix}
        cos(\theta_z) & -sin(\theta_z) \\ sin(\theta_z) & cos(\theta_z)
    \end{pmatrix}\begin{pmatrix}
        Re(S^{lib}_{x,y}(0)) \\ Imag(S^{lib}_{x,y}(0))
    \end{pmatrix} \times \left(\frac{z_{est}}{z_{lib}}\right)
    \label{eq:x-y-est}
\end{equation}

Once this step is complete, we have accounted for $x$ and $y$ rotations (\cref{eq:library-min}), $z$ rotations (\cref{eq:z-rot-est}), $x$ and $y$ translations (\cref{eq:x-y-est}), and $z$ translations (\cref{eq:z-est}). This provides a robust initial estimate when the only information available is the implant geometry and the segmentation output from the neural network. Furthermore, this can be done without any human supervision, providing a fully autonomous initialization for any pose refinement strategy, so long as the estimate is within the convergence region.

%%% Local Variables:
%%% mode: latex
%%% TeX-master: "../../../Jensen-Lit-Review"
%%% End:


\subsection{Objective Function}
\label{sec:objective-function}
In a perfect situation, our objective function would directly measure the error between our 3D model's current pose and the true pose of the object. However, if we had a-priori access to the true pose of the object, then this entire pipeline would be worthless. Thus, we must find an objective function that can act as a heuristic for the difference between true pose of our model and the current pose of our model. Our access to the segmentation output from the CNN and the ability to project the silhouette of our model quickly makes contour comparison a natural choice for an objective function. The only assumptions that we make are (1) our projective algorithm and camera definition are the same as the camera that was used to take the original fluoroscopic image and (2) our 3D model is the same 3D model that is present in the image. If these two assumptions are correct, then the alignment of the image contour and the projected contour necessarily means that our pose is correct (SYM TRAP EXCEPTION).

First, we apply a Canny edge detector (\cite{cannyComputationalApproachEdge1986}) to extract the edges from our segmentation contour, $S$, and our projected 3D model, $P$, where edges are $1$, and every other pixel is $0$. We can then iterate over each pixel and take the absolute values of the $L_1$ norm between our segmented and projected contours(\cref{eq:contour-diff}).

\begin{equation}
    J = \sum_i^{Height}\sum_j^{Width}|S_{ij} - P_{ij}|
    \label{eq:contour-diff}
\end{equation}

Unfortunately, the contours of the projected model are extremely sensitive to pose, especially when representing angles using Euler decomposition. This results in a chaotic similarity function that has an extensive amount of local minima. Past methods have overcome this by dilating the contour of the projected image (\cref{eq:dilation-erosion}) and performing the same $L_1$ optimization routine. However, a lack of an isolated contour for the fluoroscopic image still lead to a slightly noisy objective function. Our proposed method takes advantage of the segmentation output for the neural network and dilates both the segmentation contour and the projected contour for a much smoother objective function allowing for a wider search range. As our objective function is minimized, we can decrease the level of dilation to return the metric back into its original form, which most accurately describes the difference between the projection and image.

\subsection{Optimization Routine}
\label{sec:optimization-routine}
Broadly, optimization is the process of minimizing or maximizing an objective function, $f(\mathbb{R}^{n}) \rightarrow \mathbb{R}$, potentially subject to some constraints (e.g. $x \in \Omega$) \cite{audetDerivativeFreeBlackboxOptimization2017}. We formalize this below (\cref{eq:optim}). The simplest optimization problems have an analytic form of the gradient of $f$ that can be solved directly, typically by setting the first derivative to zero (e.g. least squares).

\begin{equation}
    \argmin_{x}\{f(x) : x \in \Omega\}
    \label{eq:optim}
\end{equation}

A drawback to our pipeline is that there is no analytic form of the objective function between each segmentation and projected contour; they must be resampled over a specified range in order to approximate objective function values and gradients. This defines a \emph{black box} optimization routine, which is well studied in the literature \cite{audetDerivativeFreeBlackboxOptimization2017}. In this type of function, it is not possible to use gradient-based methods to determine a minimum value for the objective function, one must use heuristics or ad-hoc methods to find the minimum location. Lipschitzian optimization offers an appealing black box optimization approach because it satisfies our need for a global search algorithm with provable convergence. First, we start with the definition of Lipschitz continuous, which places bounds on the rate of change of a function specified by some constant, called the Lipschitz constant. With a known Lipshitz constant, is is possible to find the value for the global minimum of optimization function \cite{dreisigmeyerDIRECTSEARCHMETHODS2007}.

\begin{mdframed}
    \begin{definition}[Lipschitz Continuous]
        The function $g$ is said to be Lipschitz Continuous on the set $\mathbf{X} \in \mathbb{R}^n$ if and only if there exists a scalar $K > 0$ for which
        \begin{equation*}
            \begin{aligned}
                \|g(x) - g(y)\| \le K\|x - y\|  & \\
                &\text{  for all  }& x,y \in \mathbf{X}
            \end{aligned}
        \end{equation*}
        The scalar $K$ is called the {\bf Lipschitz Constant} of $g$ relative to the set $\mathbf{X}$.
    \end{definition}
\end{mdframed}

We can illustrate this convergence with a simple example: consider the function $f(x) \rightarrow \mathbb{R}, x \in [a,b]$. If we know that the function is Lipschitz continuous, the following conditions are true on the domain $x\in[a,b]$. This corresponds to the positive and negative slopes, $K$, applied to the extrema of the domain, and the intersection, $x$ is selected as the choice for subdivision. This process is repeated, where the region is further subdivided based on the lowest value of $f(x_i)$. The iterative process is stopped once the difference between successive domain splitting is below a pre-specified global tolerance. 

\begin{equation}
    \begin{aligned}
        f(x) &\ge f(a) - K(x - a) \\
        f(x) &\ge f(b) + K(x - b)
    \end{aligned}
    \label{eq:shubert}
\end{equation}

\begin{figure}
    \begin{center}
        \includegraphics[width = 0.75\linewidth]{figures/raster/shubert-step1.png}
        \vspace{3mm}
        \includegraphics[width = 0.75\linewidth]{figures/raster/shubert-step2.png}
        \vspace{3mm}
        \includegraphics[width = 0.75\linewidth]{figures/raster/shubert-step3.png}
    \end{center}
    \label{fig:shubert}
    \caption{A visual representation of Shubert's algorthm, which finds the global minima iteratively through a repeated calculation of the intersection of two lines with slope $\pm K$. If $K$ is known, it will always find the global minimum.}
\end{figure}

While powerful, a-priori knowledge of the Lipschitz constant is needed for determining the global minima. Without it, there is no way of determining intersection points, and no way of selected new regions for subdivision and sampling. Shubert's Algorithm also has slow convergence, due to the inability to define parameters for when to explore local vs. global search. The Lipschitz constant, $K$, acts as a weight that places larger emphasis on global serach when high, and local search when low.

Fortunately, methods exist for utilizing the power of Lipschitzian optimization without the need for explicit knowledge of the Lipschitz constant \cite{jonesLipschitzianOptimizationLipschitz1993}. These can both overcome the need for an a-priori knowledge of the Lipschitz constant, as well as offer some solutions of the slow convergence by providing methods for exploiting both local and global search simultaneously to find the minimum function values.

Jones et al. \cite{jonesLipschitzianOptimizationLipschitz1993} propose a method by which the center, $c$, of a domain is sampled, rather than the endpoints. This produces the equations below (\cref{eq:direct-lipschitz}). The inequalities represent slopes $+K$ and $-K$, respectively, and provide a maximum value for the lower bound of the function at the endpoints, $a$ and $b$. The midpoints of $[a,c]$, and $[c,b]$ are then calculated and the process is then repeated (\cref{fig:direct-1D}). The power of this method is that you can determine potentially optimal regions of the domain by choosing those points along the lower convex hull of the graph plotting sub-domain size vs center point function value. The points along this hull are those that could potentially include the function minimum, and each is chosen for further sub-sampling (\cref{fig:direct-convex-hull}). Determining the convex hull is a problem well studied in the literature \cite{barberQuickhullAlgorithmConvex1996,chanOptimalOutputsensitiveConvex1996,jarvisIdentificationConvexHull1973,grahamEfficientAlgorithDetermining1972}. This elegantly mixes local vs. global search, and drastically increases the speed of convergence.


\begin{equation}
    f(x) \ge \begin{dcases}
        f(c) + K(x-c) & \text{if  } x \le c \\
        f(c) - K(x-c) & \text{if  } x \ge c
    \end{dcases}
    \label{eq:direct-lipschitz}
\end{equation}

\begin{figure}[h!]
    \begin{center}
        \includegraphics[width=0.85\linewidth]{figures/raster/direct-1D.png}
    \end{center}
    \caption{The DIviding RECTangles (DIRECT) algorithm in one dimension. It can find the global minimum of a funciton without a-priori knowledge of the Lipschitz constant. The value of the line with slpoe $\pm K$ at each of the end-points represents the theoretical minimum for the value of the function in that region. Thus, the size of the region and the value of the function at the center help the algorithm determine potentially optimal sub-regions.}
    \label{fig:direct-1D}
\end{figure}

\begin{figure}[h!]
    \begin{center}
      \includegraphics[width=0.65\linewidth]{figures/raster/direct-convex-hull.png}
    \end{center}
    \caption{The potentially optimal regions of the DIRECT algorithm are those points that lay along the lower convex hull of the scatter plot of sub-domain size vs function value at center. This is due to those regions being the locations where the maximum possible rate of change in the function might be a minimum, without any prior knowledge of the Lipschitz constant.}
    \label{fig:direct-convex-hull}
\end{figure}

This can be extended into multiple dimensions without loss of generality. First, each dimension in the domain is normalized to the range $[0,1]$, and a hypercube is created in $\mathbb{R}^D$, where $D$ is the dimension of the domain you are searching. The first iteration trisects this hybercube along an arbitrary dimension, and further iterations trisect along the largest dimension of the hypercube. We select potentially optimal hypercubes by identifying points along the lower convex hull of the graph plotting hypercube size vs center point function value.


For our purposes, we construct a hypercube along each of the 6 degrees-of-freedom that describe the pose of the implant in space, using bounds set by the user. The first epoch involves minimizing the objective function with larger bounds and a larger dilation. After all iterations have been used up, the algorithm is re-started with a smaller dilation and tighter bounds. The last epoch typically has the tightest bounds and no dilation. This pyramidal scheme offers a smooth objective function when the bounds are largest, which assists in escaping local minima, and an aggressive objective when the bounds are tight, and fewer local minima are present.

%%% Local Variables:
%%% mode: latex
%%% TeX-master: "../../../Andrew_Jensen_Dissertation"
%%% End:



\subsection{Overcoming Single-Plane Limitations}
\label{sec:single-plane-limitations}
Despite a fully autonomous solution for measuring total knee arthroplasty kinematics, there are some fundamental limitations when using monocular vision to determine the three dimensional position and orientation of an object.
\cref{sec:aim2} discusses methods for overcoming these limitations.
To date, incorporating digital ligaments into the cost function has been used to a great deal of success.
%%% Local Variables:
%%% mode: latex
%%% TeX-master: "../../../Andrew_Jensen_Dissertation"
%%% End:

