Geometric primitives, such as points, are fundamental building blocks for representing shapes and objects in computer graphics. In this section, we will discuss how points can be represented in 2D and 3D space, along with the different mathematical operations that you can perform on those points.

\subsubsection{2D and 3D Points}
\label{sec:geometric-points}
In N-dimensional space, a point is represented as a set of N scalars, each representing a magnitude in a particular direction. This can be represented mathematically as a column vector (\cref{eq:point}). In two dimensions, a point is represented with two elements, $\mathbf{x} = [x , y]^{T}$. Similarly, in three dimensions, a point is represented with three elements, $\mathbf{x} = [x , y , z]^{T}$.

\begin{equation}
    \mathbf{x} = \begin{bmatrix}
        x_1 \\ x_2 \\ \vdots \\ x_{N-1} \\ x_N
    \end{bmatrix} \in \mathbb{R}^N
    \label{eq:point}
\end{equation}

Homogeneous coordinates provide a way to represent points with an additional scale factor, $\tilde{w}$. This allows us to perform rotations and translations simultaneously and successively using matrix multiplications. Homogeneous coordinates for a point in N-dimensional space are represented as a column vector with N+1 elements (\cref{eq:homog-point}). In most model-image registration applications, the scale factor $\tilde{w}$ is set to 1. When dealing with homogeneous coordinates, $\mathbf{\tilde{x}} = \tilde{w}\mathbf{\bar{x}}$, where $\mathbf{\tilde{x}}$ is the scaled version of $\bar{\mathbf{x}}$, and $\bar{\mathbf{x}}$ is simply the original vector, $\mathbf{x}$, with a $1$ appended.

\begin{equation}
    \tilde{\mathbf{x}} = \begin{bmatrix}
        \tilde{x_1} \\ \tilde{x_2} \\ \vdots \\ \tilde{x_N} \\ \tilde{w}
    \end{bmatrix} = \tilde{w}\begin{bmatrix}
        \mathbf{x}\\ 1
    \end{bmatrix} = \tilde{w}\bar{\mathbf{x}}
    \label{eq:homog-point}
\end{equation}

\subsubsection{2D Transformations}
\label{sec:2d-transformations}
Transformations are operations that change the position, orientation, or shape of an object in 2D space. One of the most basic transformations is a translation, which moves an object by adding a displacement vector to its position (\cref{eq:translation}).

\begin{equation}
    \begin{aligned}
        \mathbf{x'} &= \begin{bmatrix}
            x  \\ y 
        \end{bmatrix} + \begin{bmatrix}
            t_x \\ t_y
        \end{bmatrix} = \mathbf{x} + \begin{bmatrix}
            t_x \\ t_y
        \end{bmatrix} = \mathbf{x} + \mathbf{t}\\
        &\text{Or, using homogeneous coordinates and matrix multiplication} \\
        &= \begin{bmatrix}
            1 & 0 & t_x \\ 0 & 1 & t_y
        \end{bmatrix} \begin{bmatrix}
            x \\ y \\ 1
        \end{bmatrix} = \begin{bmatrix}
            \mathbf{I} & \mathbf{t}
        \end{bmatrix}\bar{\mathbf{x}} \\
    \end{aligned}
    \label{eq:translation}
\end{equation}

In this equation, $\mathbf{x}$ is the original position of the object, $\mathbf{x'}$ is the transformed position, and $\mathbf{t}$ is the displacement vector that specifies the amount of translation in the $x$ and $y$ directions. Using homogeneous coordinates and matrix multiplication allows for the convenient representation of multiple transformations as a single matrix multiplication, as well as for the composition of multiple transformations (\cref{eq:mult-translation}). 

\begin{equation}
    \begin{aligned}
        \mathbf{\bar{x}}'' &= \begin{bmatrix}
            1 & 0 & t_x \\ 0 & 1 & t_y \\ 0 & 0 & 1
        \end{bmatrix} 
        \begin{bmatrix}
            1 & 0 & q_x \\ 0 & 1 & q_y \\ 0 & 0 & 1
        \end{bmatrix} \begin{bmatrix}
            x \\ y \\ 1 
        \end{bmatrix}\\
        &= \begin{bmatrix}
            x + t_x + q_x \\ y + t_y + q_y \\ 1
        \end{bmatrix}
    \end{aligned}
    \label{eq:mult-translation}
\end{equation}

The next type of 2D transformation is a rotation, which changes the orientation of an object, but not its shape (\cref{eq:2d-rot,eq:2d-rot-mat}). 

\begin{equation}
    \begin{aligned}
        \mathbf{x}' = \mathbf{Rx}
    \end{aligned}
    \label{eq:2d-rot}
\end{equation}

where
\begin{equation}
    \mathbf{R} = \begin{bmatrix}
        cos\theta & -sin \theta \\ sin \theta & cos \theta 
    \end{bmatrix}
    \label{eq:2d-rot-mat}
\end{equation}

This will rotate an object by $\theta$ in the counter clockwise direction. 

It's also possible to perform a rotation and a translation at the same time, by replacing the identity matrix $\mathbf{I}$ in Equation \ref{eq:translation} with the rotation matrix $\mathbf{R}$ from Equation \ref{eq:2d-rot-mat}. This results in a transformation that preserves lengths and angles while rotating and translating the rigid object (\cref{eq:2d-rot-trans}).

\begin{equation}
    \mathbf{x}' = \begin{bmatrix}
        \mathbf{R}_{2 \times 2} & \mathbf{t}
    \end{bmatrix} \bar{\mathbf{x}}
    \label{eq:2d-rot-trans}
\end{equation}


A scaled rotation will change the size of the object by some scaler factor, $s$ (\cref{eq:2d-scaled-rot}); this transformation preserves angles.

\begin{equation}
    \mathbf{x}' = \begin{bmatrix}
        s\mathbf{R}_{2 \times 2} & \mathbf{t}
    \end{bmatrix}\bar{\mathbf{x}}
    \label{eq:2d-scaled-rot}
\end{equation}

An affine transformation preserves parallelism, and is simply a pre-multiplication by an arbitrary $2 \times 3$ matrix (\cref{eq:2d-affine}).

\begin{equation}
    \begin{aligned}
        \mathbf{x}' &= \mathbf{A\bar{x}}\\
        &\text{where}\\
        \mathbf{A} &= \begin{bmatrix}
            a_{11} & a_{12} & a_{13} \\ a_{21} & a_{22} & a_{23}
        \end{bmatrix}
    \end{aligned}
    \label{eq:2d-affine}
\end{equation}

A projection matrix (or perspective transformation) is one that operates on homogeneous coordinates (\cref{eq:2d-projection}).

\begin{equation}
    \begin{aligned}
        \tilde{\mathbf{x}}' &= \tilde{\mathbf{H}}\tilde{\mathbf{x}}\\
    \end{aligned}
    \label{eq:2d-projection}
\end{equation}

To obtain inhomogeneous results, the resultant $\tilde{\mathbf{x}}'$ must be normalized (\cref{eq:2d-homog-norm}). A projective transformation preserves straight lines.

\begin{equation}
    \begin{aligned}
        x'&= \frac{\tilde{x}}{\tilde{w}}  &  y' &= \frac{\tilde{y}}{\tilde{w}}\\
        &= \frac{h_{11}x + h_{12}y + h_{13}}{h_{31}x + h_{32}y + h_{33}} &  &= \frac{h_{21}x + h_{22}y + h_{23}}{h_{31}x + h_{32}y + h_{33}}
    \end{aligned}
    \label{eq:2d-homog-norm}
\end{equation}


\subsubsection{3D Transformations}
\label{sec:3d-transformations}
Transformations in three dimensions are extremely similar to their two dimensional counterpart, except the dimensionality of each transformation is increased by one. However, three dimensional rotations introduce added complexity, due to multiple axes of rotation, and the non-commutativity of those rotations. We handle this by utilizing Euler angles \cite{groodJointCoordinateSystem1983}, which decomposes the final orientation of an object as a combination of rotations about the three coordinate axes. Each rotation about an axis is represented with a matrix (\cref{eq:euler-angles}), and the final rotation matrix is represented as the order of the rotations that produced it (e.g. Z-X-Y rotation).

\begin{equation}
    \begin{aligned}
        R_{x} &= \begin{bmatrix}
            1 & 0 & 0 \\ 0 & c_x & -s_x \\ 0 & s_x & c_x
        \end{bmatrix}
        &R_{y} &= \begin{bmatrix}
            c_y & 0 & s_y \\ 0 & 1 & 0 \\ -s_y & 0 & c_y
        \end{bmatrix}
        &R_{z} &= \begin{bmatrix}
            c_z & -s_z & 0 \\ s_z & c_z & 0 \\ 0 & 0 & 1
        \end{bmatrix} \\
    \end{aligned}
    \label{eq:euler-angles}
\end{equation}

Anatomically, the rotations can be chosen such that they align with the anatomic axes of the joint. This allows for congruence between the mathematical notation used to represent an object in space, and the medical notation used by orthopaedic practitioners (i.e. aligning a Z-X-Y rotation decomposition with the flexion-abduction-external rotation order used by orthopaedic surgeons).

For simplicity, we can also represent 3D rotations as an axis-angle decomposition, which describes the transformation as a single rotation about an arbitrary axis \cite{craneKinematicAnalysisRobot2008} (\cref{eq:axis-angle-matrix}). You can use trigonometric identities to determine the axis and angle given an arbitrary rotation matrix.

\begin{equation}
    \begin{aligned}
        R_{3 \times 3} &= \begin{pmatrix}
            m_x^2v + c & m_xm_yv - m_zs & m_x m_z v - m_y s \\ m_x m_y v + m_z s & m_y^2 v + c & m_y m_z v - m_x s \\ m_x m_z v - m_y s & m_y m_z v + m_x s & m_z^2 v + c
        \end{pmatrix}\\
        &\text{where} \\
        s &= sin(\theta)\\
        c &= cos(\theta) \\
        v &= (1 - c) \\
        &\text{and}\\
        \mathbf{m} &= \begin{bmatrix}
            m_x \\ m_y \\ m_z
        \end{bmatrix} \text{ is the axis of rotation}
    \end{aligned}
    \label{eq:axis-angle-matrix}
\end{equation}



\paragraph*{3D to 2D Projections}
Geometric primitives, such as points, lines, and polygons, are the building blocks of computer graphics. These basic shapes can be transformed and combined in various ways to create more complex objects and scenes. In order to represent these 3D primitives and objects in 2D image space, we must apply transformations that manipulate their position, orientation, and other properties. By using these transformations, we can create the illusion of depth and spatial relationships on a flat display. Understanding how to work with primitives and transform them is crucial for creating a wide range of visual effects and graphics in computer graphics.

In computer graphics, one of the most basic methods of projecting three-dimensional objects onto a two-dimensional image plane is an orthographic projection. This projection simply drops the depth component of the object and flattens it onto the image plane (\cref{eq:orthographic-proj}). This can be thought of as mimicking a camera with a long focal length, or when the depth of the object is shallow compared to its distance from the camera (also known as a weak perspective projection). In this equation, $\mathbf{p}$ represents a point in 3D space and $\mathbf{x}$ represents the projected point in 2D image space. The $\tilde{\cdot}$ still represents the point in homogeneous coordinates.

\begin{equation}
    \tilde{\mathbf{x}} = \begin{bmatrix}
        1 & 0 & 0 & 0 \\ 0 & 1 & 0 & 0 \\ 0 & 0 & 0 & 1
    \end{bmatrix}\tilde{\mathbf{p}}
    \label{eq:orthographic-proj}
\end{equation}

\paragraph*{Perspective Projection} 
The most common 3D-2D projection is \emph{perspective projection}, which accounts for depth perception. This can be done by scaling each point by it's $z$ position relative to the camera (Eq. \ref{eq:perspective-proj}). This necessitates using homogeneous coordinates (Eq. \ref{eq:perspective-matrix}).

\begin{equation}
    \bar{\mathbf{x}} = \mathcal{P}_z(\mathbf{p}) = \begin{bmatrix}
        x/z \\ y/z \\ 1
    \end{bmatrix}
    \label{eq:perspective-proj}
\end{equation}

\begin{equation}
    \tilde{\mathbf{x}} = \begin{bmatrix}
        1 & 0 & 0 & 0 \\ 0 & 1 & 0 & 0 \\ 0 & 0 & 1 & 0 
    \end{bmatrix}\tilde{\mathbf{p}}
    \label{eq:perspective-matrix}
\end{equation}

The perspective projection is the cornerstone of model-image registration. If a matrix mimics the fluoroscopic projective geometry in the clinic, then this offers a strong possibility of re-creating a virtual scene with the same geometry as the actual scene.

%%% Local Variables:
%%% mode: latex
%%% TeX-master: "../../../Andrew_Jensen_Dissertation.tex"
%%% End:
