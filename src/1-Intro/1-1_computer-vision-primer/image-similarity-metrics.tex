One of the key components in model image registration is image similarity. Fundamentally, this is the method of determining how well the user's synthetic image matches with the actual fluoroscopic image. The choice of similarity metric is going to be determined by many key factors such as the a-priori availability of implant/bone geometry and knowledge of the image quality and contrast. Broadly, there are two classes of image similarity when performing model-image registration: intenisty-based and feature-based.

\subsubsection{Intensity Based}
\label{sec:img-sim-intensity}
Intensity based measures are those that utilize specific pixel information in order to determine the difference between two images. This can be either a global image similarity metric, or measure the specfic regions of interest in the given image. 

A canonical difference between two images would be the p-norm separating them (Eq. \ref{eq:p-norm}), which iterates through each pixel of the two images and finds the p-norm difference each intensity for the pixel pair. Common p-norms are the $L_1$ norm (\emph{absolute intensity differences} or \emph{mean absolute difference}) \cite{kanadeStereoMatchingAlgorithm1994} ($p=1$) and the $L_{2}$, or Euclidean norm (\emph{squared intensity differences} or \emph{mean squared difference}) \cite{hannahComputerMatchingAreas1977}($p=2$).

\begin{equation}
    \|A-B\|_{p} = (\sum_{x=0}^{w}\sum_{y=0}^{h}|a_{xy}-b_{xy}|^{p})^{\frac{1}{p}}
    \label{eq:p-norm}
\end{equation}

In Equation \cref{eq:p-norm}, $A$ and $B$ are the two images being compared, $w$ and $h$ are the width and height of the images, and $a_{xy}$ and $b_{xy}$ are the intensity values at pixel $(x,y)$ in the two images, respectively.

While conceptually easy to use, the main limitation of p-norm measures is their lack of spatial information. For example, an image that has been shifted by a linear transformation would not score well using a p-norm, despite the two images containing only a minor shift, scale, or rotation. One method for overcoming this limitation is to use the cross-correlation, or sliding dot product, between images \cite{bendatRandomDataAnalysis2010,hannahComputerMatchingAreas1977} (Eq. \ref{eq:xcorr}). When used in conjunction with projective geometry, this can help locate regions of interest for a model-based registration pipeline. The cross-correlation is calculated using the following equation:

\begin{equation}
    \begin{aligned}
        (A \star B)[x,y] &= E[A_{xy} \cdot B_{x + \tau_x,y+\tau_y}] \\
        &= \sum_{\tau_x=-\infty}^{\infty}\sum_{\tau_y=-\inf}^{\infty}a_{xy}b_{x + \tau_x,y + \tau_y}
    \end{aligned}
    \label{eq:xcorr}
\end{equation}

This will have the effect of determining the regions of each image that are similar, causing the correlation function to ``light up'' at those areas in a similar way to the convolutional operation between two images. The normalized cross-correlation can also be used (Eq. \ref{eq:norm-xcorr}), which removes noise coming from each of the original images.

\begin{equation}
    \begin{aligned}
        \text{normalized cross correlation}(A,B) &= \frac{A \star B}{(A \star A)(B \star B)}
    \end{aligned}\label{eq:norm-xcorr}
\end{equation}

\subsubsection{Feature Based}
\label{sec:img-sim-feature}
Feature based image similarity metrics involve some method of determining key features in images, and using those notable features for measuring the differences between two images. These types of methods almost always involve some type of feature-extraction step, where the various features of interest are calculated and deterimined for subsequent use. The two main classes of features are \emph{keypoints} and \emph{edges}. The simplest method of keypoint detection is using a similar method to intensity-based matching, but having one of the ``images'' as a patch of the desired feature. With keypoints detected in the input image, one could determine the error of the current pose estimate by taking the Euclidean distance between all image keypoints and all projected keypoints: \cite{burtonAutomaticTrackingHealthy2021} (Eq. \ref{eq:kp-error}). With a-priori information about the keypoints, one could attach a weight to every keypoint in order to emphasize specific regions on the image and the model (Eq. \ref{eq:wkp-error})

\begin{equation}
    \begin{aligned}
        \text{Keypoint Error}= (\sum_{i = 0}^{N}(KP_{image,i} - KP_{proj,i})^2)^{\frac{1}{2}}
    \end{aligned}
    \label{eq:kp-error}
\end{equation}

\begin{equation}
    \begin{aligned}
        \text{Weighted Keypoint Error} = (\sum_{i = 0}^{N}w_{i}(KP_{image,i} - KP_{proj,i})^2)^{\frac{1}{2}}
    \end{aligned}
    \label{eq:wkp-error}
\end{equation}

Keypoints are particularly useful when there are invariant features in images and 3D models that will always be present. However, if these features will not, or cannot always be detected, then other measures must be utilized.

\paragraph*{Edges as Features}
Edges are a natural choice of feature when determining image similarity. Similar to intensity-based image similarity, the similarity between the contours of two images offers a promising metric for determining the overall similarity between two images. In model-image registration, the contours of the input image and the projected model can easily be compared, which presents a reliable scheme for measuring their similarity. When the edges are aligned, you can say that the model is \textit{properly registered} to the image. However, how can we determine when the edges are aligned?

As always, the simplest approach is to take the p-norm between the model and image contours (\cref{eq:p-norm}), where instead of taking the difference between the two original images, one is taking the difference of the edges of the images. This function will be minimized when there is complete overlap between image and model contours.


{
  \Large{MOVE SYM TRAP SOMEWHERE ELSE}
}
\subsubsection{Symmetry Traps}
Objects with rotational or mirror symmetry cause pathological solutions to many of the image similarity metrics when used for optimizing the pose of the object relative to the image. The simplest example of a symmetry trap can be posed as follows: given the shadow of a basketball, which direction was the logo facing? It is quickly apparent that this is an impossible question to answer with just the information given by the image and the 3D model. This problem arises when performing optimizing for the post of mediolaterally symmetric tibial implants. Additional information must be used to find the correct pose of the implant.

However, with the knowledge of the direction of symmetry, it is possible to determine the ``dual pose'' of the current orientation, that is, the pose that produces indistinguishable projective geometry.

\begin{mdframed}
    \begin{center}
        {\bf Algorithm for Determining the Dual-Pose of a Symmetric Object}
    \end{center}
\begin{enumerate}
    \item Determine the viewing ray from camera $\rightarrow$ object (Eq. \ref{eq:view-ray}).
    \item Determine the axis-angle ($m,\theta$) rotation between the viewing ray and the symmetric axis of the object (Eq. \ref{eq:angle-between}, \ref{eq:perp-axis}).
    \item Rotate the object $-2\theta$ about the same axis, reflecting the rotation about the viewing ray (Eq. \ref{eq:equiv-axis-angle}).
    \item The final orientation of the object is exactly the ``dual pose'', producing indistinguishable projective geometry (Eq. \ref{eq:rotation-matrix-mult}).
\end{enumerate}
\end{mdframed}

\begin{equation}
    \begin{aligned}
        \text{If $T$ is the homogenous}& \text{ transformation matrix describing the object} \\
        \vec{v}' &= T_{1:3,4} \\
        \vec{v} & = \frac{\vec{v}'}{\|\vec{v}'\|}
    \end{aligned}
    \label{eq:view-ray}
\end{equation}


We can use trigonometry to determine the angle (Eq. \ref{eq:angle-between}) and perpindicular axis (Eq. \ref{eq:perp-axis}) between two vectors. For our example, we use the normalized viewing ray and the z-axis (symmetric axis) as the two vectors.
\begin{equation}
    \begin{aligned}
        cos(\theta) &= \vec{v} \cdot \vec{z} \\
        &\rightarrow \\
        \theta &= arccos(\vec{v} \cdot \vec{z})
    \end{aligned}
    \label{eq:angle-between}
\end{equation}

\begin{equation}
    \begin{aligned}
        \vec{m} = \frac{\vec{v} \times \vec{z}}{\|\vec{v} \times \vec{z}\|}
    \end{aligned}
    \label{eq:perp-axis}
\end{equation}

Then, we can build a rotation matrix using an axis and an angle \cite{craneKinematicAnalysisRobot2008}, (Eq. \ref{eq:equiv-axis-angle}).

\begin{equation}
    \begin{aligned}
        c &= cos(-2\theta)\\
        s &= sin(-2\theta) \\
        q &= -cos(-2\theta)\\
        R_{3 \times 3} &= \begin{pmatrix}
            m_x^2v + c & m_xm_yv - m_zs & m_x m_z v - m_y s \\ m_x m_y v + m_z s & m_y^2 v + c & m_y m_z v - m_x s \\ m_x m_z v - m_y s & m_y m_z v + m_x s & m_z^2 v + c
        \end{pmatrix}
    \end{aligned}
    \label{eq:equiv-axis-angle}
\end{equation}

Then, we obtain the final transformation matrix describing the dual pose of the object by a post-multiplication of this rotation matrix.

\begin{equation}
    \begin{aligned}
        T_{dual} = T_{orig} * \begin{pmatrix}
            R_{3 \times 3} & \vec{0}_{3 \times 1} \\ \vec{0}_{1 \times 3} & 1
        \end{pmatrix}
    \end{aligned}
    \label{eq:rotation-matrix-mult}
\end{equation}

Given these two matrices, further exploration can determine which is the correct pose, though this will have to be done using information not directly present in the image contours.

%%% Local Variables:
%%% mode: latex
%%% TeX-master: "../../../Jensen-Lit-Review"
%%% End:
