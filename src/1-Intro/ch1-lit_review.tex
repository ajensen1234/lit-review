\chapter{Introduction}
Total Knee Arthroplasty (TKA) is a standard procedure for alleviating symptoms related to osteoarthritis in the knee. In 2018, orthopaedic surgeons performed more than 715,000 TKA operations in the United States \cite{agencyforhealthcareresearchandqualityHCUPFastStats}. This number is projected to increase to 3.48 million by 2030 \cite{kurtzProjectionsPrimaryRevision2007} due to an aging population and increased obesity rates. While TKA largely relieves symptomatic osteoarthritis, roughly 20\% of TKA patients express postoperative dissatisfaction, citing mechanical limitations, pain, and instability as the leading causes \cite{bakerRolePainFunction2007,bournePatientSatisfactionTotal2010,scottPredictingDissatisfactionFollowing2010}. Standard methods of musculoskeletal diagnosis cannot quantify the dynamic state of the joint, either pre- or post-operatively; clinicians must rely on static imaging (radiography, MRI, CT) or qualitative mechanical tests to determine the condition of the affected joint, and these tests cannot easily be performed during weight-bearing or dynamic movement when most pain symptoms occur. Unfortunately, most of the tools used to quantify 3D dynamic motion are substantially affected by soft-tissue artifacts \cite{gaoInvestigationSoftTissue2008,stagniQuantificationSoftTissue2005,linEffectsSoftTissue2016}, are prohibitively time-consuming or expensive \cite{daemsValidationThreedimensionalTotal2016a}, or cannot be performed with equipment available at most hospitals.

Model-image registration is a process where a 3D model is aligned to match an object’s projection in an image \cite{brownSurveyImageRegistration1992}. Researchers have performed model-image registration using single-plane fluoroscopic or flat-panel imaging since the 1990s. Early methods used pre-computed distance maps \cite{lavalleeRecoveringPositionOrientation1995,zuffiModelbasedMethodReconstruction1999}, or shape libraries \cite{banksAccurateMeasurementThreedimensional1996,wallaceAnalysisThreedimensionalMovement1980,wallaceEfficientThreedimensionalAircraft1980} to match the projection of a 3D implant model to its projection in a radiographic image. With increasing computational capabilities, methods that iteratively compared implant projections to images were possible \cite{mahfouzRobustMethodRegistration2003,floodAutomatedRegistration3D2018,loweFittingParameterizedThreedimensional1991}. Most model-image registration methods provide sufficient accuracy for clinical joint assessment applications, including natural and replaced knees \cite{banksVivoKinematicsCruciateretaining1997,banks2003HapPaul2004,komistekVivoFluoroscopicAnalysis2003,burtonAutomaticTrackingHealthy2021}, natural and replaced shoulders \cite{kijimaVivo3dimensionalAnalysis2015,mahfouzVivoDeterminationDynamics,matsukiVivo3dimensionalAnalysis2011,sugiComparingVivoThreedimensional2021}, and extremities \cite{cenniKinematicsThreeComponents2012,cenniFunctionalPerformanceTotal2013,deaslaSixDOFVivo2006}. One of the main benefits of this single-plane approach is that suitable images can be acquired with equipment found in most hospitals. The main impediment to implementing this approach into a standard clinical workflow is the time and expense of human operators to supervise the model-image registration process. These methods require either (1) an initial pose estimate \cite{floodAutomatedRegistration3D2018,loweFittingParameterizedThreedimensional1991}, (2) a pre-segmented contour of the implant in the image \cite{brownSurveyImageRegistration1992,lavalleeRecoveringPositionOrientation1995}, or (3) a human operator to assist the optimization routine out of local minima \cite{mahfouzRobustMethodRegistration2003}. Each of these requirements makes model-image registration methods impractical for clinical use. Even state-of-the-art model-image registration techniques \cite{floodAutomatedRegistration3D2018} require human initialization or segmentation to perform adequately.

Machine learning algorithms automate the process of analytical model building, utilizing specific algorithms to fit a series of inputs to their respective outputs. Neural networks are a subset of machine learning algorithms that utilize artificial neurons inspired by the human brain’s connections \cite{marrEarlyProcessingVisual1976}. These networks have shown a great deal of success in many computer vision tasks, such as segmentation \cite{chanHistoSegNetSemanticSegmentation2019,wangDeepHighResolutionRepresentation2020,ronnebergerUNetConvolutionalNetworks2015}, pose estimation \cite{wuDeepGraphPose2020,kendallGeometricLossFunctions2017}, and classification \cite{krizhevskyImageNetClassificationDeep2017,qiPointNetDeepHierarchical2017,qiPointNetDeepLearning2017}. These capabilities might remove the need for human supervision from TKA model-image registration. Therefore, we propose a three-stage data analysis pipeline where a convolutional neural network (CNN) is used to segment, or identify, the pixels belonging to either a femoral or tibial component. Then, an initial pose estimate is generated comparing the segmented implant contour to a pre-computed shape library. Lastly, the initial pose estimate serves as the starting point for a Lipschitzian optimizer that aligns the contours of a 3D implant model to the contour of the CNN-segmented image.

\section{Background}
\subsection{Current Ortho Exams}
\input{src/1-Intro/1-1_background/ortho-exams.tex}
\subsection{Fluoroscopy}
\input{src/1-Intro/1-1_background/fluoroscopy.tex}
\subsection{Kinematics from Fluoroscopy}
\input{src/1-Intro/1-1_background/kinematics-from-fluoro.tex}